%========================================
% preamble.tex — Preâmbulo unificado
% Livro: Introdução à Assimilação de Dados
% Autor: João Gerd Zell de Mattos
%========================================

%----------------------------
% Codificação, língua, fonte
%----------------------------
\usepackage[utf8]{inputenc}   % entrada UTF-8
\usepackage[T1]{fontenc}      % saída com acentuação correta
\usepackage[brazil]{babel}    % hifenização/nomes em pt-BR
\usepackage{lmodern}          % fonte Latin Modern
\usepackage{microtype}        % microtipografia (melhora a legibilidade)

%----------------------------
% Página e tipografia
%----------------------------
\usepackage{geometry}         % margens e formato de página
\geometry{a4paper, margin=2.5cm}
\usepackage{setspace}         % espaçamento entre linhas
\onehalfspacing
\usepackage{parskip}          % parágrafo com espaço em vez de indentação

%----------------------------
% Cores e hyperlinks
%----------------------------
\usepackage{xcolor}
\definecolor{primary}{RGB}{20, 92, 160}     % azul científico
\definecolor{secondary}{RGB}{0, 134, 114}   % verde água
\definecolor{accent}{RGB}{180, 60, 60}      % vermelho suave
\definecolor{textgray}{gray}{0.15}
% cores adicionais harmônicas
\definecolor{neutral}{gray}{0.25}           % cinza neutro
\definecolor{violet}{RGB}{110, 75, 160}     % violeta acadêmico
\definecolor{teal}{RGB}{30, 140, 140}       % verde-azulado elegante
\definecolor{orange}{RGB}{230,140, 40}      % laranja para observações
\color{textgray}

\usepackage{hyperref}                   % hyperlinks no PDF
\hypersetup{
  colorlinks=true,
  linkcolor=primary,
  citecolor=secondary,
  urlcolor=primary,
  pdfauthor={João Gerd Zell de Mattos},
  pdftitle={Introdução à Assimilação de Dados},
  pdfcreator={LaTeX},
  pdfsubject={Assimilação de Dados},
  pdfkeywords={assimilação de dados, var, enkf, híbrido, previsão numérica}
}
\usepackage{bookmark}                   % melhora marcadores do PDF (carregar DEPOIS de hyperref)

% cotações tipográficas — recomendado quando usar biblatex/estilo autor-ano
\usepackage{csquotes}

%----------------------------
% Matemática e SI
%----------------------------
\usepackage{amsmath,amssymb,mathtools} % matemática base
\usepackage{bm}                        % negrito em símbolos (vetores/matrizes)
\usepackage{physics}                   % notação conveniente (derivadas, etc.)
\usepackage{siunitx}                   % unidades SI e números
\sisetup{
  detect-all,
  per-mode=symbol,
  separate-uncertainty=true,
  exponent-product=\cdot,
  output-decimal-marker={,}  % pt-BR
}

% Operadores úteis
\DeclareMathOperator*{\argmin}{arg\,min}
\DeclareMathOperator*{\argmax}{arg\,max}

%----------------------------
% Figuras, tabelas, legendas
%----------------------------
\usepackage{graphicx}       % inclusão de figuras
\usepackage{booktabs}       % tabelas bonitas (\toprule, \midrule, \bottomrule)
\usepackage{caption}        % personalização de legendas
\usepackage{subcaption}     % subfiguras
\captionsetup{labelfont=bf}

%----------------------------
% TikZ / PGFPlots (figuras 100% LaTeX)
%----------------------------
\usepackage{tikz}
\usetikzlibrary{
  arrows.meta,positioning,calc,fit,
  decorations.pathmorphing,
  shapes.geometric,
  shadows, fadings
}
\tikzset{
  >={Stealth[length=2.0mm]},
  box/.style   ={draw, rounded corners=4pt, inner sep=6pt, thick, fill=primary!5},
  flow/.style  ={->, thick},
  circ/.style  ={draw, circle, minimum size=9mm, thick, fill=secondary!10},
  note/.style  ={draw, rounded corners=3pt, inner sep=5pt, dashed, color=secondary},
}

\usepackage{pgfplots}       % gráficos de dados
\pgfplotsset{compat=1.18}   % define compatibilidade estável do pgfplots

% (Opcional) macros próprias para setas em arco — carregar somente se for usado globalmente
% ==============================================================
%  Macro e estilo do arco em seta (\arcarrow)
%  >> Inclua este arquivo com: % ==============================================================
%  Macro e estilo do arco em seta (\arcarrow)
%  >> Inclua este arquivo com: % ==============================================================
%  Macro e estilo do arco em seta (\arcarrow)
%  >> Inclua este arquivo com: \input{fig/arcarrow.tex}
% ==============================================================
% --- TikZ: bibliotecas necessárias ---
\usetikzlibrary{decorations.text,positioning, shadows}
\usetikzlibrary{arrows.meta,calc} % (calc para coordenadas polares/expressões)
% --------------------------------------------------------------
% Estilo do texto desenhado AO LONGO DO ARCO (interno às setas)
% --------------------------------------------------------------
\newcommand*{\mytextstyle}{\sffamily\footnotesize\bfseries\color{black!85}}

% --------------------------------------------------------------
% MACRO PRINCIPAL: \arcarrow
% Desenha UMA seta em arco com texto curvado.
%
% Assinatura:
%   \arcarrow{rin}{rmid}{rout}{angIni}{angFim}{tip}{tikz-options}{texto}
%
% - rin, rmid, rout: raios interno, médio (texto) e externo da seta.
% - angIni, angFim:  ângulos em graus (orientação padrão do TikZ).
% - tip:              avanço (em graus) no raio médio para formar a “ponta”.
% - tikz-options:     opções passadas ao \fill (fill, draw, espessura etc.).
% - texto:            string desenhada ao longo do arco médio.
% --------------------------------------------------------------
\newcommand{\arcarrow}[8]{%
  % Normalização numérica (evita expansão prematura)
  \pgfmathsetmacro{\rin}{#1}
  \pgfmathsetmacro{\rmid}{#2}
  \pgfmathsetmacro{\rout}{#3}
  \pgfmathsetmacro{\astart}{#4}
  \pgfmathsetmacro{\aend}{#5}
  \pgfmathsetmacro{\atip}{#6}

  % Shape preenchido (cunha + ponta)
  \fill[#7]
    (\astart:\rin) arc (\astart:\aend:\rin)   % arco interno
    -- (\aend+\atip:\rmid)                    % ponta no raio médio
    -- (\aend:\rout) arc (\aend:\astart:\rout)% arco externo (volta)
    -- (\astart+\atip:\rmid) -- cycle;        % fecha a cunha no raio médio

  % Texto ao longo do arco médio (opcional; remova se não quiser texto interno)
  \path[
    decoration={text along path, text={|\mytextstyle|#8},
                text align=center, raise=-0.3ex},
    decorate
  ] (\astart+\atip:\rmid) arc (\astart+\atip:\aend+\atip:\rmid);
}



% ==============================================================
% --- TikZ: bibliotecas necessárias ---
\usetikzlibrary{decorations.text,positioning, shadows}
\usetikzlibrary{arrows.meta,calc} % (calc para coordenadas polares/expressões)
% --------------------------------------------------------------
% Estilo do texto desenhado AO LONGO DO ARCO (interno às setas)
% --------------------------------------------------------------
\newcommand*{\mytextstyle}{\sffamily\footnotesize\bfseries\color{black!85}}

% --------------------------------------------------------------
% MACRO PRINCIPAL: \arcarrow
% Desenha UMA seta em arco com texto curvado.
%
% Assinatura:
%   \arcarrow{rin}{rmid}{rout}{angIni}{angFim}{tip}{tikz-options}{texto}
%
% - rin, rmid, rout: raios interno, médio (texto) e externo da seta.
% - angIni, angFim:  ângulos em graus (orientação padrão do TikZ).
% - tip:              avanço (em graus) no raio médio para formar a “ponta”.
% - tikz-options:     opções passadas ao \fill (fill, draw, espessura etc.).
% - texto:            string desenhada ao longo do arco médio.
% --------------------------------------------------------------
\newcommand{\arcarrow}[8]{%
  % Normalização numérica (evita expansão prematura)
  \pgfmathsetmacro{\rin}{#1}
  \pgfmathsetmacro{\rmid}{#2}
  \pgfmathsetmacro{\rout}{#3}
  \pgfmathsetmacro{\astart}{#4}
  \pgfmathsetmacro{\aend}{#5}
  \pgfmathsetmacro{\atip}{#6}

  % Shape preenchido (cunha + ponta)
  \fill[#7]
    (\astart:\rin) arc (\astart:\aend:\rin)   % arco interno
    -- (\aend+\atip:\rmid)                    % ponta no raio médio
    -- (\aend:\rout) arc (\aend:\astart:\rout)% arco externo (volta)
    -- (\astart+\atip:\rmid) -- cycle;        % fecha a cunha no raio médio

  % Texto ao longo do arco médio (opcional; remova se não quiser texto interno)
  \path[
    decoration={text along path, text={|\mytextstyle|#8},
                text align=center, raise=-0.3ex},
    decorate
  ] (\astart+\atip:\rmid) arc (\astart+\atip:\aend+\atip:\rmid);
}



% ==============================================================
% --- TikZ: bibliotecas necessárias ---
\usetikzlibrary{decorations.text,positioning, shadows}
\usetikzlibrary{arrows.meta,calc} % (calc para coordenadas polares/expressões)
% --------------------------------------------------------------
% Estilo do texto desenhado AO LONGO DO ARCO (interno às setas)
% --------------------------------------------------------------
\newcommand*{\mytextstyle}{\sffamily\footnotesize\bfseries\color{black!85}}

% --------------------------------------------------------------
% MACRO PRINCIPAL: \arcarrow
% Desenha UMA seta em arco com texto curvado.
%
% Assinatura:
%   \arcarrow{rin}{rmid}{rout}{angIni}{angFim}{tip}{tikz-options}{texto}
%
% - rin, rmid, rout: raios interno, médio (texto) e externo da seta.
% - angIni, angFim:  ângulos em graus (orientação padrão do TikZ).
% - tip:              avanço (em graus) no raio médio para formar a “ponta”.
% - tikz-options:     opções passadas ao \fill (fill, draw, espessura etc.).
% - texto:            string desenhada ao longo do arco médio.
% --------------------------------------------------------------
\newcommand{\arcarrow}[8]{%
  % Normalização numérica (evita expansão prematura)
  \pgfmathsetmacro{\rin}{#1}
  \pgfmathsetmacro{\rmid}{#2}
  \pgfmathsetmacro{\rout}{#3}
  \pgfmathsetmacro{\astart}{#4}
  \pgfmathsetmacro{\aend}{#5}
  \pgfmathsetmacro{\atip}{#6}

  % Shape preenchido (cunha + ponta)
  \fill[#7]
    (\astart:\rin) arc (\astart:\aend:\rin)   % arco interno
    -- (\aend+\atip:\rmid)                    % ponta no raio médio
    -- (\aend:\rout) arc (\aend:\astart:\rout)% arco externo (volta)
    -- (\astart+\atip:\rmid) -- cycle;        % fecha a cunha no raio médio

  % Texto ao longo do arco médio (opcional; remova se não quiser texto interno)
  \path[
    decoration={text along path, text={|\mytextstyle|#8},
                text align=center, raise=-0.3ex},
    decorate
  ] (\astart+\atip:\rmid) arc (\astart+\atip:\aend+\atip:\rmid);
}




%----------------------------
% Glossário e índice
%----------------------------
\usepackage[acronym,shortcuts,toc]{glossaries} % acrônimos + entrada no sumário
% Observação: \makeglossaries e %======================================================================
% Arquivo: glossary.tex
% Livro: Introdução à Assimilação de Dados
% Autor: João Gerd Zell de Mattos
%
% Descrição:
%   Contém todas as siglas, abreviações, símbolos e termos do livro.
%   Deve ser incluído após o comando \makeglossaries no main.tex.
%======================================================================

%----------------------------------------------------------------------
% SIGLAS E ABREVIATURAS
%----------------------------------------------------------------------
\newacronym{ad}{AD}{Assimilação de Dados}
\newacronym{obana}{OBAN}{Análise Objetiva}
\newacronym{var}{VAR}{Método Variacional}
\newacronym{3dvar}{3DVar}{Three-Dimensional Variational Method}
\newacronym{4dvar}{4DVar}{Four-Dimensional Variational Method}
\newacronym{enkf}{EnKF}{Ensemble Kalman Filter}
\newacronym{letkf}{LETKF}{Local Ensemble Transform Kalman Filter}
\newacronym{enkf-mean}{EnKF-Mean}{Filtro de Kalman de Conjunto com Atualização do Estado Médio}
\newacronym{enkf-pert}{EnKF-Pert}{Filtro de Kalman de Conjunto com Atualização das Perturbações}
\newacronym{envar}{EnVar}{Método Híbrido Ensemble-Variacional}
\newacronym{bmat}{\textbf{B}}{Matriz de Covariância de Erro de Background}
\newacronym{rmat}{\textbf{R}}{Matriz de Covariância de Erro de Observação}
\newacronym{hmat}{\textbf{H}}{Operador de Observação Linearizado}
\newacronym{kmat}{\textbf{K}}{Ganho de Kalman}
\newacronym{jcost}{\textit{J}}{Função Custo da Assimilação}
\newacronym{bam}{BAM}{Brazilian Atmospheric Model}
\newacronym{gsi}{GSI}{Gridpoint Statistical Interpolation}
\newacronym{jedi}{JEDI}{Joint Effort for Data Assimilation Integration}
\newacronym{monan}{MONAN}{Modelo para Previsão dos Oceanos, Superfícies Terrestres e Atmosfera}
\newacronym{cptec}{CPTEC}{Centro de Previsão de Tempo e Estudos Climáticos}
\newacronym{inpe}{INPE}{Instituto Nacional de Pesquisas Espaciais}
\newacronym{ncep}{NCEP}{National Centers for Environmental Prediction}
\newacronym{ncar}{NCAR}{National Center for Atmospheric Research}
\newacronym{ucan}{UCAN}{Unified Community for Assimilation and Nowcasting}
\newacronym{satobs}{SATOBS}{Observações de Satélite}
\newacronym{convobs}{CONVOBS}{Observações Convencionais}
\newacronym{obsproc}{OBSPROC}{Pré-processamento de Observações}
\newacronym{enkfens}{Ens}{Ensemble (Conjunto de Membros)}
\newacronym{varbc}{VarBC}{Bias Correction Variacional}

%----------------------------------------------------------------------
% SÍMBOLOS E NOTAÇÃO
%----------------------------------------------------------------------
\newglossaryentry{state-vector}{
  name={\ensuremath{\vect{x}}},
  description={Vetor de estado do modelo, contendo as variáveis prognósticas a serem atualizadas pela assimilação.}
}

\newglossaryentry{obs-vector}{
  name={\ensuremath{\vect{y}}},
  description={Vetor de observações, representando os dados medidos ou inferidos do sistema terrestre.}
}

\newglossaryentry{b-matrix}{
  name={\ensuremath{\mat{B}}},
  description={Matriz de covariância de erro de background, que representa incertezas associadas à previsão de primeira estimativa.}
}

\newglossaryentry{r-matrix}{
  name={\ensuremath{\mat{R}}},
  description={Matriz de covariância de erro de observação, descrevendo as incertezas instrumentais e representativas.}
}

\newglossaryentry{gain-matrix}{
  name={\ensuremath{\mat{K}}},
  description={Matriz de ganho de Kalman, responsável por ponderar a influência das observações na atualização do estado.}
}

\newglossaryentry{cost-function}{
  name={\ensuremath{\mathcal{J}}},
  description={Função custo que mede o desvio entre o estado analisado e o background e entre as observações e suas estimativas.}
}

\newglossaryentry{h-operator}{
  name={\ensuremath{\mat{H}}},
  description={Operador de observação (ou seu linearizado), que transforma o espaço do modelo no espaço das observações.}
}

%----------------------------------------------------------------------
% TERMOS CONCEITUAIS (exemplo de glossário descritivo)
%----------------------------------------------------------------------
\newglossaryentry{analise}{
  name={Análise},
  description={Resultado do processo de assimilação de dados — um campo que combina previsões de modelo com observações, ponderadas por suas incertezas.}
}

\newglossaryentry{background}{
  name={Background},
  description={Estimativa inicial (geralmente uma previsão de curto prazo) usada como ponto de partida na assimilação.}
}
\newglossaryentry{chute-inicial}{
  name={Chute inicial},
  description={Estimativa de curto prazo do estado atmosférico utilizada como ponto de partida no processo de assimilação de dados. 
  Também chamada de \textit{first guess} ou \textit{background}, é obtida normalmente pela integração anterior do modelo numérico e representa a melhor estimativa disponível antes da incorporação das observações mais recentes.}
}
\newglossaryentry{blue}{
  name={Melhor estimativa estatística (BLUE)},
  description={Acrônimo de \textit{Best Linear Unbiased Estimator}, ou melhor estimador linear não-viesado. 
  Representa a estimativa que minimiza a variância do erro entre o estado analisado e o verdadeiro, sob as hipóteses de linearidade, gaussianidade e não-viesamento dos erros. 
  Constitui a base teórica da Interpolação Ótima e do Filtro de Kalman, formando o princípio estatístico fundamental da assimilação de dados.}
}

\newglossaryentry{forecast}{
  name={Previsão},
  description={Resultado da integração do modelo a partir da análise ou de um estado inicial fornecido.}
}

\newglossaryentry{variacional}{
  name={Método Variacional},
  description={Técnica de assimilação que encontra o estado ótimo minimizando uma função custo em um espaço contínuo.}
}

\newglossaryentry{ensemble}{
  name={Ensemble},
  description={Conjunto de múltiplas previsões do modelo com pequenas perturbações iniciais, usado para representar incertezas.}
}

%======================================================================
% FIM DO ARQUIVO
%======================================================================

 ficam no main.tex
\usepackage{makeidx}        % índice remissivo (\makeindex no main.tex)

%----------------------------
% Bibliografia (biblatex)
%----------------------------
% Mantém o carregamento aqui; o \addbibresource{...} fica no main.tex.
\usepackage[backend=biber,style=authoryear,maxbibnames=10,uniquename=false,natbib=true]{biblatex}

%----------------------------
% Estilo de títulos / epígrafe / listas
%----------------------------
\usepackage{titlesec}       % personalização de títulos (\chapter, \section, ...)
\usepackage{epigraph}       % epígrafes elegantes no início de capítulos
\setlength{\epigraphwidth}{0.7\textwidth}
\setlength{\epigraphrule}{0pt}

\usepackage{enumitem}       % listas com controle de espaçamento/labels
\setlist{itemsep=2pt, topsep=4pt}

%----------------------------
% Referências “inteligentes”
%----------------------------
\usepackage[nameinlink,noabbrev]{cleveref}
% Uso: \cref{fig:exemplo}, \Cref{eq:custo}, etc.

%----------------------------
% Listagens de código
%----------------------------
\usepackage{listings}
\lstdefinestyle{code}{
  basicstyle=\ttfamily\small,
  numbers=left,
  numberstyle=\tiny,
  stepnumber=1,
  numbersep=6pt,
  showstringspaces=false,
  breaklines=true,
  frame=single,
  rulecolor=\color{black!20},
  keywordstyle=\color{primary}\bfseries,
  commentstyle=\color{secondary!80!black}\itshape,
  stringstyle=\color{accent!90!black},
  tabsize=2
}
\lstset{style=code}

%======================================================
% CAIXAS EDITORIAIS UNIFICADAS (tcolorbox)
%======================================================
\usepackage[most]{tcolorbox}
\usepackage{fontawesome5}

\tcbset{
  enhanced, sharp corners=downhill, arc=2pt, boxsep=3pt,
  left=6pt, right=6pt, top=6pt, bottom=6pt, breakable
}

% =============== FABRICA DE CAIXAS (macro genérica) ===============
% Uso: \MakeBoxEnv{nome}{Titulo}{corBase}{icone}
\newcommand{\MakeBoxEnv}[4]{%
  % #1 = nome do ambiente (sem acento), ex: definicao
  % #2 = título (pode ter acento),    ex: Definição
  % #3 = cor base,                    ex: primary
  % #4 = ícone (fontawesome),         ex: \faBook
  \newtcolorbox{#1}[1][]{
    enhanced,
    colback=#3!6,                   % fundo suave
    colframe=#3!80!black,           % borda
    colbacktitle=#3!80!black,       % barra superior
    coltitle=white,                 % cor do título
    title={#4\ \textbf{#2}},        % título + ícone
    fonttitle=\bfseries\sffamily,   % estilo do título
    left=6pt, right=6pt, top=6pt, bottom=6pt,
    attach boxed title to top left={yshift=-2mm, xshift=2mm},
    boxed title style={
      sharp corners,
      rounded corners=northwest,
      rounded corners=northeast,
      boxrule=0pt,
      colframe=#3!80!black,
      colback=#3!80!black,
    },
    drop shadow,                    % sombra leve
    #1                              % opções extras
  }%
}

% ================== CAIXAS EDITORIAIS ==================
% já tens: definicao (mantém para coerência)
\MakeBoxEnv{definicao}{Definição}{primary}{\faBook}

% pedido: destaque, nota, alerta, exemplo
\MakeBoxEnv{destaque}{Destaque}{primary}{\faHighlighter}
\MakeBoxEnv{nota}{Nota}{secondary}{\faStickyNote}
\MakeBoxEnv{alerta}{Atenção}{accent}{\faExclamationTriangle}
\MakeBoxEnv{exemplo}{Exemplo}{neutral}{\faFlask}

% extras úteis em livro
\MakeBoxEnv{observacao}{Observação}{orange}{\faInfoCircle}
\MakeBoxEnv{dica}{Dica}{teal}{\faLightbulb}
\MakeBoxEnv{exercicio}{Exercício}{violet}{\faDumbbell}
\MakeBoxEnv{prova}{Prova}{accent}{\faPenFancy} % para demonstrações matemáticas
\MakeBoxEnv{comentario}{Comentário}{neutral}{\faCommentDots}
\MakeBoxEnv{sintese}{Síntese}{violet}{\faProjectDiagram}
%----------------------------
% Teoremas e afins (pt-BR)
%----------------------------
\usepackage{amsthm}
\theoremstyle{plain}
\newtheorem{teorema}{Teorema}[chapter]
\newtheorem{proposicao}[teorema]{Proposição}
\newtheorem{lema}[teorema]{Lema}
\theoremstyle{definition}
\newtheorem{definicao}[teorema]{Definição}
\newtheorem{exercicio}[teorema]{Exercício}
\theoremstyle{remark}
\newtheorem{observacao}[teorema]{Observação}

%----------------------------
% Cabeçalho/Rodapé
%----------------------------
\usepackage{fancyhdr}
\pagestyle{fancy}
\fancyhf{}
\fancyhead[LE,RO]{\thepage}
\fancyhead[LO]{\nouppercase{\rightmark}}
\fancyhead[RE]{\nouppercase{\leftmark}}
\renewcommand{\headrulewidth}{0.4pt}

%----------------------------
% Toggle rascunho/final
%----------------------------
\newif\ifdraft
\draftfalse % mude para \drafttrue durante a escrita (marcações úteis abaixo)

\ifdraft
  \usepackage[inline]{showlabels} % mostra labels ao lado (debug)
  \renewcommand{\baselinestretch}{1.2}
\fi

%----------------------------
% Comandos úteis p/ Assimilação de Dados
%----------------------------
% Vetores e matrizes (em negrito)
\newcommand{\vect}[1]{\bm{#1}}
\newcommand{\mat}[1]{\bm{#1}}

% Estados e observações
\newcommand{\xb}{\vect{x}_b}     % background
\newcommand{\xa}{\vect{x}_a}     % análise
\newcommand{\xo}{\vect{x}}       % estado genérico
\newcommand{\yobs}{\vect{y}}     % observações

% Operadores e matrizes
\newcommand{\Hop}{\mat{H}}       % operador de observação
\newcommand{\Kmat}{\mat{K}}      % ganho de Kalman
\newcommand{\Bmat}{\mat{B}}      % covariância de background
\newcommand{\Rmat}{\mat{R}}      % covariância de observação
\newcommand{\Pmat}{\mat{P}}      % genérica (Ex.: erro de previsão)
\newcommand{\J}{\mathcal{J}}     % função custo
\newcommand{\grad}{\nabla}       % gradiente
\newcommand{\T}{^{\mathsf T}}    % transposta

% Variação incremental (x' etc.)
\newcommand{\dx}{\delta \vect{x}}
\newcommand{\dy}{\delta \vect{y}}

% Conveniências
\newcommand{\norm}[1]{\left\lVert #1 \right\rVert}
\newcommand{\abs}[1]{\left\lvert #1 \right\rvert}

% Numeração por capítulo
\numberwithin{equation}{chapter}
\numberwithin{figure}{chapter}
\numberwithin{table}{chapter}


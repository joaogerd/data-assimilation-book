\documentclass[border=0pt]{standalone}
%\documentclass[12pt,a4paper]{article}
% --- Fonte/idioma (ok com pdfLaTeX ou XeLaTeX) ---
\usepackage[T1]{fontenc}
\usepackage[utf8]{inputenc}
\usepackage[brazil]{babel}
\usepackage{lmodern}
\usepackage{microtype}

% --- TikZ ---
\usepackage{tikz}
\usetikzlibrary{arrows.meta,positioning,calc,shadings}

% --- Tamanho físico da página A4 + margem zero ---
\newdimen\coverW \coverW=210mm
\newdimen\coverH \coverH=297mm
\paperwidth=\coverW
\paperheight=\coverH
\usepackage[paperwidth=\paperwidth,paperheight=\paperheight,margin=0pt]{geometry}

% --- Paleta ---
\definecolor{coverBgA}{HTML}{0E2341}
\definecolor{coverBgB}{HTML}{123E7A}
\definecolor{accentC}{HTML}{29ABE2}
\definecolor{accentG}{HTML}{1ABC9C}
\definecolor{textWhite}{RGB}{245,245,247}
\definecolor{softLine}{RGB}{200,220,245}

% --- Metadados ---
\newcommand{\BookTitleA}{Introdução à Assimilação de Dados}
%\newcommand{\BookTitleB}{Dados}
\newcommand{\BookSubtitle}{Fundamentos, prática e perspectivas}
\newcommand{\BookAuthor}{João Gerd Zell de Mattos}

\begin{document}
\begin{tikzpicture} % <<< sem overlay

  % 0) FUNDO SÓLIDO (diagnóstico): se isso não aparecer, nada mais aparecerá
  \fill[coverBgA] (0,0) rectangle (\coverW,\coverH);

  % 1) Gradiente por cima (se quiser o efeito original)
  %    -> Se suspeitar do 'shade' no seu driver, comente estas 2 linhas.
  \path[shade, left color=coverBgA, right color=coverBgB]
    (0,0) rectangle (\coverW,\coverH);

  % 2) Anéis
  \pgfmathsetlengthmacro{\cx}{0.8*\coverW}
  \pgfmathsetlengthmacro{\cy}{0.7*\coverH}
  \foreach \r in {25mm,45mm,70mm,95mm}{
    \draw[line width=0.6pt, draw=softLine, opacity=0.25] (\cx,\cy) circle (\r);
  }
  \pgfmathsetlengthmacro{\cx}{0.2*\coverW}
  \pgfmathsetlengthmacro{\cy}{0.2*\coverH}
  \foreach \r in {25mm,45mm,70mm,95mm}{
    \draw[line width=0.6pt, draw=softLine, opacity=0.25] (\cx,\cy) circle (\r);
  }


  % 3) Rede leve (pode comentar se ficar pesado)
  \begin{scope}[opacity=0.08]
    \foreach \x in {0,15,...,210}{
      \draw[textWhite!60] (\x mm,0) -- (\x mm,\coverH);
    }
    \foreach \y in {0,15,...,297}{
      \draw[textWhite!60] (0,\y mm) -- (\coverW,\y mm);
    }
  \end{scope}

%  % 4) Ciclo de assimilação
%  \pgfmathsetlengthmacro{\rc}{38mm}
%  \pgfmathsetlengthmacro{\cx}{0.72*\coverW}
%  \pgfmathsetlengthmacro{\cy}{0.60*\coverH}
%
%  \node[draw=accentC, fill=accentC!10, text=textWhite, rounded corners=3pt,
%        minimum width=28mm, align=center, font=\footnotesize]
%    (n1) at ($(\cx,\cy) + (  0:\rc)$) {Previsão\\(modelo)};
%  \node[draw=accentC, fill=accentC!10, text=textWhite, rounded corners=3pt,
%        minimum width=30mm, align=center, font=\footnotesize]
%    (n2) at ($(\cx,\cy) + ( 90:\rc)$) {Observações\\(satélite/superfície)};
%  \node[draw=accentC, fill=accentC!10, text=textWhite, rounded corners=3pt,
%        minimum width=28mm, align=center, font=\footnotesize]
%    (n3) at ($(\cx,\cy) + (180:\rc)$) {Análise\\(assimilação)};
%  \node[draw=accentC, fill=accentC!10, text=textWhite, rounded corners=3pt,
%        minimum width=28mm, align=center, font=\footnotesize]
%    (n4) at ($(\cx,\cy) + (270:\rc)$) {Integração\\(ciclo)};
%
%  \tikzset{cycarr/.style={-Latex, line width=0.9pt, draw=accentG}}
%  \draw[cycarr] (n1) to[bend left=18] (n2);
%  \draw[cycarr] (n2) to[bend left=18] (n3);
%  \draw[cycarr] (n3) to[bend left=18] (n4);
%  \draw[cycarr] (n4) to[bend left=18] (n1);
%
%  \draw[draw=accentG!60, line width=1pt, opacity=0.35] (\cx,\cy) circle (\rc+10mm);

  % 5) Título
  \node[anchor=center, text=textWhite, align=center]
    at (0.12*\coverW,0.62*\coverH) {%
      {\fontsize{34pt}{38pt}\selectfont\bfseries \BookTitleA}
    };
%
%  % 6) Subtítulo (barra)
%  \pgfmathsetlengthmacro{\bx}{0.12*\coverW}
%  \pgfmathsetlengthmacro{\by}{0.49*\coverH}
%  \pgfmathsetlengthmacro{\bw}{0.58*\coverW}
%  \pgfmathsetlengthmacro{\bh}{14mm}
%  \fill[accentC!20, rounded corners=2mm] (\bx,\by) rectangle (\bx+\bw,\by+\bh);
%  \node[anchor=west, text=textWhite, align=left, font=\large]
%    at (\bx+6mm,\by+0.5*\bh) {\BookSubtitle};

  % 7) Autor
  \node[anchor=west, text=textWhite!92, align=left, font=\normalsize]
    at (0.12*\coverW,0.18*\coverH) {\BookAuthor};

  % 8) Faixa inferior
  \fill[accentG!20, opacity=0.25] (0,0) rectangle (\coverW,10mm);

  % DEBUG opcional: enxerga a área útil
  % \draw[red,very thick] (0,0) rectangle (\coverW,\coverH);

\end{tikzpicture}
\end{document}


%======================================================================
% Arquivo: glossary.tex
% Livro: Introdução à Assimilação de Dados
% Autor: João Gerd Zell de Mattos
%
% Descrição:
%   Contém todas as siglas, abreviações, símbolos e termos do livro.
%   Deve ser incluído após o comando \makeglossaries no main.tex.
%======================================================================

%----------------------------------------------------------------------
% SIGLAS E ABREVIATURAS
%----------------------------------------------------------------------
\newacronym{ad}{AD}{Assimilação de Dados}
\newacronym{obana}{OBAN}{Análise Objetiva}
\newacronym{var}{VAR}{Método Variacional}
\newacronym{3dvar}{3DVar}{Three-Dimensional Variational Method}
\newacronym{4dvar}{4DVar}{Four-Dimensional Variational Method}
\newacronym{enkf}{EnKF}{Ensemble Kalman Filter}
\newacronym{letkf}{LETKF}{Local Ensemble Transform Kalman Filter}
\newacronym{enkf-mean}{EnKF-Mean}{Filtro de Kalman de Conjunto com Atualização do Estado Médio}
\newacronym{enkf-pert}{EnKF-Pert}{Filtro de Kalman de Conjunto com Atualização das Perturbações}
\newacronym{envar}{EnVar}{Método Híbrido Ensemble-Variacional}
\newacronym{bmat}{\textbf{B}}{Matriz de Covariância de Erro de Background}
\newacronym{rmat}{\textbf{R}}{Matriz de Covariância de Erro de Observação}
\newacronym{hmat}{\textbf{H}}{Operador de Observação Linearizado}
\newacronym{kmat}{\textbf{K}}{Ganho de Kalman}
\newacronym{jcost}{\textit{J}}{Função Custo da Assimilação}
\newacronym{bam}{BAM}{Brazilian Atmospheric Model}
\newacronym{gsi}{GSI}{Gridpoint Statistical Interpolation}
\newacronym{jedi}{JEDI}{Joint Effort for Data Assimilation Integration}
\newacronym{monan}{MONAN}{Modelo para Previsão dos Oceanos, Superfícies Terrestres e Atmosfera}
\newacronym{cptec}{CPTEC}{Centro de Previsão de Tempo e Estudos Climáticos}
\newacronym{inpe}{INPE}{Instituto Nacional de Pesquisas Espaciais}
\newacronym{ncep}{NCEP}{National Centers for Environmental Prediction}
\newacronym{ncar}{NCAR}{National Center for Atmospheric Research}
\newacronym{ucan}{UCAN}{Unified Community for Assimilation and Nowcasting}
\newacronym{satobs}{SATOBS}{Observações de Satélite}
\newacronym{convobs}{CONVOBS}{Observações Convencionais}
\newacronym{obsproc}{OBSPROC}{Pré-processamento de Observações}
\newacronym{enkfens}{Ens}{Ensemble (Conjunto de Membros)}
\newacronym{varbc}{VarBC}{Bias Correction Variacional}

%----------------------------------------------------------------------
% SÍMBOLOS E NOTAÇÃO
%----------------------------------------------------------------------
\newglossaryentry{state-vector}{
  name={\ensuremath{\vect{x}}},
  description={Vetor de estado do modelo, contendo as variáveis prognósticas a serem atualizadas pela assimilação.}
}

\newglossaryentry{obs-vector}{
  name={\ensuremath{\vect{y}}},
  description={Vetor de observações, representando os dados medidos ou inferidos do sistema terrestre.}
}

\newglossaryentry{b-matrix}{
  name={\ensuremath{\mat{B}}},
  description={Matriz de covariância de erro de background, que representa incertezas associadas à previsão de primeira estimativa.}
}

\newglossaryentry{r-matrix}{
  name={\ensuremath{\mat{R}}},
  description={Matriz de covariância de erro de observação, descrevendo as incertezas instrumentais e representativas.}
}

\newglossaryentry{gain-matrix}{
  name={\ensuremath{\mat{K}}},
  description={Matriz de ganho de Kalman, responsável por ponderar a influência das observações na atualização do estado.}
}

\newglossaryentry{cost-function}{
  name={\ensuremath{\mathcal{J}}},
  description={Função custo que mede o desvio entre o estado analisado e o background e entre as observações e suas estimativas.}
}

\newglossaryentry{h-operator}{
  name={\ensuremath{\mat{H}}},
  description={Operador de observação (ou seu linearizado), que transforma o espaço do modelo no espaço das observações.}
}

%----------------------------------------------------------------------
% TERMOS CONCEITUAIS (exemplo de glossário descritivo)
%----------------------------------------------------------------------
\newglossaryentry{analise}{
  name={Análise},
  description={Resultado do processo de assimilação de dados — um campo que combina previsões de modelo com observações, ponderadas por suas incertezas.}
}

\newglossaryentry{background}{
  name={Background},
  description={Estimativa inicial (geralmente uma previsão de curto prazo) usada como ponto de partida na assimilação.}
}

\newglossaryentry{forecast}{
  name={Previsão},
  description={Resultado da integração do modelo a partir da análise ou de um estado inicial fornecido.}
}

\newglossaryentry{variacional}{
  name={Método Variacional},
  description={Técnica de assimilação que encontra o estado ótimo minimizando uma função custo em um espaço contínuo.}
}

\newglossaryentry{ensemble}{
  name={Ensemble},
  description={Conjunto de múltiplas previsões do modelo com pequenas perturbações iniciais, usado para representar incertezas.}
}

%======================================================================
% FIM DO ARQUIVO
%======================================================================


%======================================================================
% Livro: Introdução à Assimilação de Dados
% Autor: João Gerd Zell de Mattos
% Arquivo principal (main.tex)
%
% Compilação recomendada:
%   latexmk -pdf -interaction=nonstopmode main.tex
%
% Estrutura recomendada:
%   src/
%     ├── preamble.tex
%     ├── glossary.tex
%     ├── bib/references.bib
%     ├── chapters/ch01.tex ... ch11.tex
%======================================================================

\documentclass[12pt,a4paper,oneside]{book}

%----------------------------------------------------------------------
% Importa o preâmbulo unificado (pacotes, cores, estilos, comandos AD)
%----------------------------------------------------------------------
%========================================
% preamble.tex — Preâmbulo unificado
% Livro: Introdução à Assimilação de Dados
% Autor: João Gerd Zell de Mattos
%========================================

%----------------------------
% Codificação, língua, fonte
%----------------------------
\usepackage[utf8]{inputenc}   % entrada UTF-8
\usepackage[T1]{fontenc}      % saída com acentuação correta
\usepackage[brazil]{babel}    % hifenização/nomes em pt-BR
\usepackage{lmodern}          % fonte Latin Modern
\usepackage{microtype}        % microtipografia (melhora a legibilidade)

%----------------------------
% Página e tipografia
%----------------------------
\usepackage{geometry}         % margens e formato de página
\geometry{a4paper, margin=2.5cm}
\usepackage{setspace}         % espaçamento entre linhas
\onehalfspacing
\usepackage{parskip}          % parágrafo com espaço em vez de indentação

%----------------------------
% Cores e hyperlinks
%----------------------------
\usepackage{xcolor}
\definecolor{primary}{RGB}{20, 92, 160}     % azul científico
\definecolor{secondary}{RGB}{0, 134, 114}   % verde água
\definecolor{accent}{RGB}{180, 60, 60}      % vermelho suave
\definecolor{textgray}{gray}{0.15}
% cores adicionais harmônicas
\definecolor{neutral}{gray}{0.25}           % cinza neutro
\definecolor{violet}{RGB}{110, 75, 160}     % violeta acadêmico
\definecolor{teal}{RGB}{30, 140, 140}       % verde-azulado elegante
\definecolor{orange}{RGB}{230,140, 40}      % laranja para observações
\color{textgray}

\usepackage{hyperref}                   % hyperlinks no PDF
\hypersetup{
  colorlinks=true,
  linkcolor=primary,
  citecolor=secondary,
  urlcolor=primary,
  pdfauthor={João Gerd Zell de Mattos},
  pdftitle={Introdução à Assimilação de Dados},
  pdfcreator={LaTeX},
  pdfsubject={Assimilação de Dados},
  pdfkeywords={assimilação de dados, var, enkf, híbrido, previsão numérica}
}
\usepackage{bookmark}                   % melhora marcadores do PDF (carregar DEPOIS de hyperref)

% cotações tipográficas — recomendado quando usar biblatex/estilo autor-ano
\usepackage{csquotes}

%----------------------------
% Matemática e SI
%----------------------------
\usepackage{amsmath,amssymb,mathtools} % matemática base
\usepackage{bm}                        % negrito em símbolos (vetores/matrizes)
\usepackage{physics}                   % notação conveniente (derivadas, etc.)
\usepackage{siunitx}                   % unidades SI e números
\sisetup{
  detect-all,
  per-mode=symbol,
  separate-uncertainty=true,
  exponent-product=\cdot,
  output-decimal-marker={,}  % pt-BR
}

% Operadores úteis
\DeclareMathOperator*{\argmin}{arg\,min}
\DeclareMathOperator*{\argmax}{arg\,max}

%----------------------------
% Figuras, tabelas, legendas
%----------------------------
\usepackage{graphicx}       % inclusão de figuras
\usepackage{booktabs}       % tabelas bonitas (\toprule, \midrule, \bottomrule)
\usepackage{caption}        % personalização de legendas
\usepackage{subcaption}     % subfiguras
\captionsetup{labelfont=bf}

%----------------------------
% TikZ / PGFPlots (figuras 100% LaTeX)
%----------------------------
\usepackage{tikz}
\usetikzlibrary{
  arrows.meta,positioning,calc,fit,
  decorations.pathmorphing,
  shapes.geometric,
  shadows, fadings
}
\tikzset{
  >={Stealth[length=2.0mm]},
  box/.style   ={draw, rounded corners=4pt, inner sep=6pt, thick, fill=primary!5},
  flow/.style  ={->, thick},
  circ/.style  ={draw, circle, minimum size=9mm, thick, fill=secondary!10},
  note/.style  ={draw, rounded corners=3pt, inner sep=5pt, dashed, color=secondary},
}

\usepackage{pgfplots}       % gráficos de dados
\pgfplotsset{compat=1.18}   % define compatibilidade estável do pgfplots

% (Opcional) macros próprias para setas em arco — carregar somente se for usado globalmente
% ==============================================================
%  Macro e estilo do arco em seta (\arcarrow)
%  >> Inclua este arquivo com: % ==============================================================
%  Macro e estilo do arco em seta (\arcarrow)
%  >> Inclua este arquivo com: \input{fig/arcarrow.tex}
% ==============================================================
% --- TikZ: bibliotecas necessárias ---
\usetikzlibrary{decorations.text,positioning, shadows}
\usetikzlibrary{arrows.meta,calc} % (calc para coordenadas polares/expressões)
% --------------------------------------------------------------
% Estilo do texto desenhado AO LONGO DO ARCO (interno às setas)
% --------------------------------------------------------------
\newcommand*{\mytextstyle}{\sffamily\footnotesize\bfseries\color{black!85}}

% --------------------------------------------------------------
% MACRO PRINCIPAL: \arcarrow
% Desenha UMA seta em arco com texto curvado.
%
% Assinatura:
%   \arcarrow{rin}{rmid}{rout}{angIni}{angFim}{tip}{tikz-options}{texto}
%
% - rin, rmid, rout: raios interno, médio (texto) e externo da seta.
% - angIni, angFim:  ângulos em graus (orientação padrão do TikZ).
% - tip:              avanço (em graus) no raio médio para formar a “ponta”.
% - tikz-options:     opções passadas ao \fill (fill, draw, espessura etc.).
% - texto:            string desenhada ao longo do arco médio.
% --------------------------------------------------------------
\newcommand{\arcarrow}[8]{%
  % Normalização numérica (evita expansão prematura)
  \pgfmathsetmacro{\rin}{#1}
  \pgfmathsetmacro{\rmid}{#2}
  \pgfmathsetmacro{\rout}{#3}
  \pgfmathsetmacro{\astart}{#4}
  \pgfmathsetmacro{\aend}{#5}
  \pgfmathsetmacro{\atip}{#6}

  % Shape preenchido (cunha + ponta)
  \fill[#7]
    (\astart:\rin) arc (\astart:\aend:\rin)   % arco interno
    -- (\aend+\atip:\rmid)                    % ponta no raio médio
    -- (\aend:\rout) arc (\aend:\astart:\rout)% arco externo (volta)
    -- (\astart+\atip:\rmid) -- cycle;        % fecha a cunha no raio médio

  % Texto ao longo do arco médio (opcional; remova se não quiser texto interno)
  \path[
    decoration={text along path, text={|\mytextstyle|#8},
                text align=center, raise=-0.3ex},
    decorate
  ] (\astart+\atip:\rmid) arc (\astart+\atip:\aend+\atip:\rmid);
}



% ==============================================================
% --- TikZ: bibliotecas necessárias ---
\usetikzlibrary{decorations.text,positioning, shadows}
\usetikzlibrary{arrows.meta,calc} % (calc para coordenadas polares/expressões)
% --------------------------------------------------------------
% Estilo do texto desenhado AO LONGO DO ARCO (interno às setas)
% --------------------------------------------------------------
\newcommand*{\mytextstyle}{\sffamily\footnotesize\bfseries\color{black!85}}

% --------------------------------------------------------------
% MACRO PRINCIPAL: \arcarrow
% Desenha UMA seta em arco com texto curvado.
%
% Assinatura:
%   \arcarrow{rin}{rmid}{rout}{angIni}{angFim}{tip}{tikz-options}{texto}
%
% - rin, rmid, rout: raios interno, médio (texto) e externo da seta.
% - angIni, angFim:  ângulos em graus (orientação padrão do TikZ).
% - tip:              avanço (em graus) no raio médio para formar a “ponta”.
% - tikz-options:     opções passadas ao \fill (fill, draw, espessura etc.).
% - texto:            string desenhada ao longo do arco médio.
% --------------------------------------------------------------
\newcommand{\arcarrow}[8]{%
  % Normalização numérica (evita expansão prematura)
  \pgfmathsetmacro{\rin}{#1}
  \pgfmathsetmacro{\rmid}{#2}
  \pgfmathsetmacro{\rout}{#3}
  \pgfmathsetmacro{\astart}{#4}
  \pgfmathsetmacro{\aend}{#5}
  \pgfmathsetmacro{\atip}{#6}

  % Shape preenchido (cunha + ponta)
  \fill[#7]
    (\astart:\rin) arc (\astart:\aend:\rin)   % arco interno
    -- (\aend+\atip:\rmid)                    % ponta no raio médio
    -- (\aend:\rout) arc (\aend:\astart:\rout)% arco externo (volta)
    -- (\astart+\atip:\rmid) -- cycle;        % fecha a cunha no raio médio

  % Texto ao longo do arco médio (opcional; remova se não quiser texto interno)
  \path[
    decoration={text along path, text={|\mytextstyle|#8},
                text align=center, raise=-0.3ex},
    decorate
  ] (\astart+\atip:\rmid) arc (\astart+\atip:\aend+\atip:\rmid);
}




%----------------------------
% Glossário e índice
%----------------------------
\usepackage[acronym,shortcuts,toc]{glossaries} % acrônimos + entrada no sumário
% Observação: \makeglossaries e %======================================================================
% Arquivo: glossary.tex
% Livro: Introdução à Assimilação de Dados
% Autor: João Gerd Zell de Mattos
%
% Descrição:
%   Contém todas as siglas, abreviações, símbolos e termos do livro.
%   Deve ser incluído após o comando \makeglossaries no main.tex.
%======================================================================

%----------------------------------------------------------------------
% SIGLAS E ABREVIATURAS
%----------------------------------------------------------------------
\newacronym{ad}{AD}{Assimilação de Dados}
\newacronym{obana}{OBAN}{Análise Objetiva}
\newacronym{var}{VAR}{Método Variacional}
\newacronym{3dvar}{3DVar}{Three-Dimensional Variational Method}
\newacronym{4dvar}{4DVar}{Four-Dimensional Variational Method}
\newacronym{enkf}{EnKF}{Ensemble Kalman Filter}
\newacronym{letkf}{LETKF}{Local Ensemble Transform Kalman Filter}
\newacronym{enkf-mean}{EnKF-Mean}{Filtro de Kalman de Conjunto com Atualização do Estado Médio}
\newacronym{enkf-pert}{EnKF-Pert}{Filtro de Kalman de Conjunto com Atualização das Perturbações}
\newacronym{envar}{EnVar}{Método Híbrido Ensemble-Variacional}
\newacronym{bmat}{\textbf{B}}{Matriz de Covariância de Erro de Background}
\newacronym{rmat}{\textbf{R}}{Matriz de Covariância de Erro de Observação}
\newacronym{hmat}{\textbf{H}}{Operador de Observação Linearizado}
\newacronym{kmat}{\textbf{K}}{Ganho de Kalman}
\newacronym{jcost}{\textit{J}}{Função Custo da Assimilação}
\newacronym{bam}{BAM}{Brazilian Atmospheric Model}
\newacronym{gsi}{GSI}{Gridpoint Statistical Interpolation}
\newacronym{jedi}{JEDI}{Joint Effort for Data Assimilation Integration}
\newacronym{monan}{MONAN}{Modelo para Previsão dos Oceanos, Superfícies Terrestres e Atmosfera}
\newacronym{cptec}{CPTEC}{Centro de Previsão de Tempo e Estudos Climáticos}
\newacronym{inpe}{INPE}{Instituto Nacional de Pesquisas Espaciais}
\newacronym{ncep}{NCEP}{National Centers for Environmental Prediction}
\newacronym{ncar}{NCAR}{National Center for Atmospheric Research}
\newacronym{ucan}{UCAN}{Unified Community for Assimilation and Nowcasting}
\newacronym{satobs}{SATOBS}{Observações de Satélite}
\newacronym{convobs}{CONVOBS}{Observações Convencionais}
\newacronym{obsproc}{OBSPROC}{Pré-processamento de Observações}
\newacronym{enkfens}{Ens}{Ensemble (Conjunto de Membros)}
\newacronym{varbc}{VarBC}{Bias Correction Variacional}

%----------------------------------------------------------------------
% SÍMBOLOS E NOTAÇÃO
%----------------------------------------------------------------------
\newglossaryentry{state-vector}{
  name={\ensuremath{\vect{x}}},
  description={Vetor de estado do modelo, contendo as variáveis prognósticas a serem atualizadas pela assimilação.}
}

\newglossaryentry{obs-vector}{
  name={\ensuremath{\vect{y}}},
  description={Vetor de observações, representando os dados medidos ou inferidos do sistema terrestre.}
}

\newglossaryentry{b-matrix}{
  name={\ensuremath{\mat{B}}},
  description={Matriz de covariância de erro de background, que representa incertezas associadas à previsão de primeira estimativa.}
}

\newglossaryentry{r-matrix}{
  name={\ensuremath{\mat{R}}},
  description={Matriz de covariância de erro de observação, descrevendo as incertezas instrumentais e representativas.}
}

\newglossaryentry{gain-matrix}{
  name={\ensuremath{\mat{K}}},
  description={Matriz de ganho de Kalman, responsável por ponderar a influência das observações na atualização do estado.}
}

\newglossaryentry{cost-function}{
  name={\ensuremath{\mathcal{J}}},
  description={Função custo que mede o desvio entre o estado analisado e o background e entre as observações e suas estimativas.}
}

\newglossaryentry{h-operator}{
  name={\ensuremath{\mat{H}}},
  description={Operador de observação (ou seu linearizado), que transforma o espaço do modelo no espaço das observações.}
}

%----------------------------------------------------------------------
% TERMOS CONCEITUAIS (exemplo de glossário descritivo)
%----------------------------------------------------------------------
\newglossaryentry{analise}{
  name={Análise},
  description={Resultado do processo de assimilação de dados — um campo que combina previsões de modelo com observações, ponderadas por suas incertezas.}
}

\newglossaryentry{background}{
  name={Background},
  description={Estimativa inicial (geralmente uma previsão de curto prazo) usada como ponto de partida na assimilação.}
}
\newglossaryentry{chute-inicial}{
  name={Chute inicial},
  description={Estimativa de curto prazo do estado atmosférico utilizada como ponto de partida no processo de assimilação de dados. 
  Também chamada de \textit{first guess} ou \textit{background}, é obtida normalmente pela integração anterior do modelo numérico e representa a melhor estimativa disponível antes da incorporação das observações mais recentes.}
}
\newglossaryentry{blue}{
  name={Melhor estimativa estatística (BLUE)},
  description={Acrônimo de \textit{Best Linear Unbiased Estimator}, ou melhor estimador linear não-viesado. 
  Representa a estimativa que minimiza a variância do erro entre o estado analisado e o verdadeiro, sob as hipóteses de linearidade, gaussianidade e não-viesamento dos erros. 
  Constitui a base teórica da Interpolação Ótima e do Filtro de Kalman, formando o princípio estatístico fundamental da assimilação de dados.}
}

\newglossaryentry{forecast}{
  name={Previsão},
  description={Resultado da integração do modelo a partir da análise ou de um estado inicial fornecido.}
}

\newglossaryentry{variacional}{
  name={Método Variacional},
  description={Técnica de assimilação que encontra o estado ótimo minimizando uma função custo em um espaço contínuo.}
}

\newglossaryentry{ensemble}{
  name={Ensemble},
  description={Conjunto de múltiplas previsões do modelo com pequenas perturbações iniciais, usado para representar incertezas.}
}

%======================================================================
% FIM DO ARQUIVO
%======================================================================

 ficam no main.tex
\usepackage{makeidx}        % índice remissivo (\makeindex no main.tex)

%----------------------------
% Bibliografia (biblatex)
%----------------------------
% Mantém o carregamento aqui; o \addbibresource{...} fica no main.tex.
\usepackage[backend=biber,style=authoryear,maxbibnames=10,uniquename=false,natbib=true]{biblatex}

%----------------------------
% Estilo de títulos / epígrafe / listas
%----------------------------
\usepackage{titlesec}       % personalização de títulos (\chapter, \section, ...)
\usepackage{epigraph}       % epígrafes elegantes no início de capítulos
\setlength{\epigraphwidth}{0.7\textwidth}
\setlength{\epigraphrule}{0pt}

\usepackage{enumitem}       % listas com controle de espaçamento/labels
\setlist{itemsep=2pt, topsep=4pt}

%----------------------------
% Referências “inteligentes”
%----------------------------
\usepackage[nameinlink,noabbrev]{cleveref}
% Uso: \cref{fig:exemplo}, \Cref{eq:custo}, etc.

%----------------------------
% Listagens de código
%----------------------------
\usepackage{listings}
\lstdefinestyle{code}{
  basicstyle=\ttfamily\small,
  numbers=left,
  numberstyle=\tiny,
  stepnumber=1,
  numbersep=6pt,
  showstringspaces=false,
  breaklines=true,
  frame=single,
  rulecolor=\color{black!20},
  keywordstyle=\color{primary}\bfseries,
  commentstyle=\color{secondary!80!black}\itshape,
  stringstyle=\color{accent!90!black},
  tabsize=2
}
\lstset{style=code}

%======================================================
% CAIXAS EDITORIAIS UNIFICADAS (tcolorbox)
%======================================================
\usepackage[most]{tcolorbox}
\usepackage{fontawesome5}

\tcbset{
  enhanced, sharp corners=downhill, arc=2pt, boxsep=3pt,
  left=6pt, right=6pt, top=6pt, bottom=6pt, breakable
}

% =============== FABRICA DE CAIXAS (macro genérica) ===============
% Uso: \MakeBoxEnv{nome}{Titulo}{corBase}{icone}
\newcommand{\MakeBoxEnv}[4]{%
  % #1 = nome do ambiente (sem acento), ex: definicao
  % #2 = título (pode ter acento),    ex: Definição
  % #3 = cor base,                    ex: primary
  % #4 = ícone (fontawesome),         ex: \faBook
  \newtcolorbox{#1}[1][]{
    enhanced,
    colback=#3!6,                   % fundo suave
    colframe=#3!80!black,           % borda
    colbacktitle=#3!80!black,       % barra superior
    coltitle=white,                 % cor do título
    title={#4\ \textbf{#2}},        % título + ícone
    fonttitle=\bfseries\sffamily,   % estilo do título
    left=6pt, right=6pt, top=6pt, bottom=6pt,
    attach boxed title to top left={yshift=-2mm, xshift=2mm},
    boxed title style={
      sharp corners,
      rounded corners=northwest,
      rounded corners=northeast,
      boxrule=0pt,
      colframe=#3!80!black,
      colback=#3!80!black,
    },
    drop shadow,                    % sombra leve
    #1                              % opções extras
  }%
}

% ================== CAIXAS EDITORIAIS ==================
% já tens: definicao (mantém para coerência)
\MakeBoxEnv{definicao}{Definição}{primary}{\faBook}

% pedido: destaque, nota, alerta, exemplo
\MakeBoxEnv{destaque}{Destaque}{primary}{\faHighlighter}
\MakeBoxEnv{nota}{Nota}{secondary}{\faStickyNote}
\MakeBoxEnv{alerta}{Atenção}{accent}{\faExclamationTriangle}
\MakeBoxEnv{exemplo}{Exemplo}{neutral}{\faFlask}

% extras úteis em livro
\MakeBoxEnv{observacao}{Observação}{orange}{\faInfoCircle}
\MakeBoxEnv{dica}{Dica}{teal}{\faLightbulb}
\MakeBoxEnv{exercicio}{Exercício}{violet}{\faDumbbell}
\MakeBoxEnv{prova}{Prova}{accent}{\faPenFancy} % para demonstrações matemáticas
\MakeBoxEnv{comentario}{Comentário}{neutral}{\faCommentDots}
\MakeBoxEnv{sintese}{Síntese}{violet}{\faProjectDiagram}
%----------------------------
% Teoremas e afins (pt-BR)
%----------------------------
\usepackage{amsthm}
\theoremstyle{plain}
\newtheorem{teorema}{Teorema}[chapter]
\newtheorem{proposicao}[teorema]{Proposição}
\newtheorem{lema}[teorema]{Lema}
\theoremstyle{definition}
\newtheorem{definicao}[teorema]{Definição}
\newtheorem{exercicio}[teorema]{Exercício}
\theoremstyle{remark}
\newtheorem{observacao}[teorema]{Observação}

%----------------------------
% Cabeçalho/Rodapé
%----------------------------
\usepackage{fancyhdr}
\pagestyle{fancy}
\fancyhf{}
\fancyhead[LE,RO]{\thepage}
\fancyhead[LO]{\nouppercase{\rightmark}}
\fancyhead[RE]{\nouppercase{\leftmark}}
\renewcommand{\headrulewidth}{0.4pt}

%----------------------------
% Toggle rascunho/final
%----------------------------
\newif\ifdraft
\draftfalse % mude para \drafttrue durante a escrita (marcações úteis abaixo)

\ifdraft
  \usepackage[inline]{showlabels} % mostra labels ao lado (debug)
  \renewcommand{\baselinestretch}{1.2}
\fi

%----------------------------
% Comandos úteis p/ Assimilação de Dados
%----------------------------
% Vetores e matrizes (em negrito)
\newcommand{\vect}[1]{\bm{#1}}
\newcommand{\mat}[1]{\bm{#1}}

% Estados e observações
\newcommand{\xb}{\vect{x}_b}     % background
\newcommand{\xa}{\vect{x}_a}     % análise
\newcommand{\xo}{\vect{x}}       % estado genérico
\newcommand{\yobs}{\vect{y}}     % observações

% Operadores e matrizes
\newcommand{\Hop}{\mat{H}}       % operador de observação
\newcommand{\Kmat}{\mat{K}}      % ganho de Kalman
\newcommand{\Bmat}{\mat{B}}      % covariância de background
\newcommand{\Rmat}{\mat{R}}      % covariância de observação
\newcommand{\Pmat}{\mat{P}}      % genérica (Ex.: erro de previsão)
\newcommand{\J}{\mathcal{J}}     % função custo
\newcommand{\grad}{\nabla}       % gradiente
\newcommand{\T}{^{\mathsf T}}    % transposta

% Variação incremental (x' etc.)
\newcommand{\dx}{\delta \vect{x}}
\newcommand{\dy}{\delta \vect{y}}

% Conveniências
\newcommand{\norm}[1]{\left\lVert #1 \right\rVert}
\newcommand{\abs}[1]{\left\lvert #1 \right\rvert}

% Numeração por capítulo
\numberwithin{equation}{chapter}
\numberwithin{figure}{chapter}
\numberwithin{table}{chapter}



%----------------------------------------------------------------------
% Bibliografia e glossário (carregados após o preâmbulo)
%----------------------------------------------------------------------
\addbibresource{bib/references.bib} % Caminho relativo
\makeglossaries
%======================================================================
% Arquivo: glossary.tex
% Livro: Introdução à Assimilação de Dados
% Autor: João Gerd Zell de Mattos
%
% Descrição:
%   Contém todas as siglas, abreviações, símbolos e termos do livro.
%   Deve ser incluído após o comando \makeglossaries no main.tex.
%======================================================================

%----------------------------------------------------------------------
% SIGLAS E ABREVIATURAS
%----------------------------------------------------------------------
\newacronym{ad}{AD}{Assimilação de Dados}
\newacronym{obana}{OBAN}{Análise Objetiva}
\newacronym{var}{VAR}{Método Variacional}
\newacronym{3dvar}{3DVar}{Three-Dimensional Variational Method}
\newacronym{4dvar}{4DVar}{Four-Dimensional Variational Method}
\newacronym{enkf}{EnKF}{Ensemble Kalman Filter}
\newacronym{letkf}{LETKF}{Local Ensemble Transform Kalman Filter}
\newacronym{enkf-mean}{EnKF-Mean}{Filtro de Kalman de Conjunto com Atualização do Estado Médio}
\newacronym{enkf-pert}{EnKF-Pert}{Filtro de Kalman de Conjunto com Atualização das Perturbações}
\newacronym{envar}{EnVar}{Método Híbrido Ensemble-Variacional}
\newacronym{bmat}{\textbf{B}}{Matriz de Covariância de Erro de Background}
\newacronym{rmat}{\textbf{R}}{Matriz de Covariância de Erro de Observação}
\newacronym{hmat}{\textbf{H}}{Operador de Observação Linearizado}
\newacronym{kmat}{\textbf{K}}{Ganho de Kalman}
\newacronym{jcost}{\textit{J}}{Função Custo da Assimilação}
\newacronym{bam}{BAM}{Brazilian Atmospheric Model}
\newacronym{gsi}{GSI}{Gridpoint Statistical Interpolation}
\newacronym{jedi}{JEDI}{Joint Effort for Data Assimilation Integration}
\newacronym{monan}{MONAN}{Modelo para Previsão dos Oceanos, Superfícies Terrestres e Atmosfera}
\newacronym{cptec}{CPTEC}{Centro de Previsão de Tempo e Estudos Climáticos}
\newacronym{inpe}{INPE}{Instituto Nacional de Pesquisas Espaciais}
\newacronym{ncep}{NCEP}{National Centers for Environmental Prediction}
\newacronym{ncar}{NCAR}{National Center for Atmospheric Research}
\newacronym{ucan}{UCAN}{Unified Community for Assimilation and Nowcasting}
\newacronym{satobs}{SATOBS}{Observações de Satélite}
\newacronym{convobs}{CONVOBS}{Observações Convencionais}
\newacronym{obsproc}{OBSPROC}{Pré-processamento de Observações}
\newacronym{enkfens}{Ens}{Ensemble (Conjunto de Membros)}
\newacronym{varbc}{VarBC}{Bias Correction Variacional}

%----------------------------------------------------------------------
% SÍMBOLOS E NOTAÇÃO
%----------------------------------------------------------------------
\newglossaryentry{state-vector}{
  name={\ensuremath{\vect{x}}},
  description={Vetor de estado do modelo, contendo as variáveis prognósticas a serem atualizadas pela assimilação.}
}

\newglossaryentry{obs-vector}{
  name={\ensuremath{\vect{y}}},
  description={Vetor de observações, representando os dados medidos ou inferidos do sistema terrestre.}
}

\newglossaryentry{b-matrix}{
  name={\ensuremath{\mat{B}}},
  description={Matriz de covariância de erro de background, que representa incertezas associadas à previsão de primeira estimativa.}
}

\newglossaryentry{r-matrix}{
  name={\ensuremath{\mat{R}}},
  description={Matriz de covariância de erro de observação, descrevendo as incertezas instrumentais e representativas.}
}

\newglossaryentry{gain-matrix}{
  name={\ensuremath{\mat{K}}},
  description={Matriz de ganho de Kalman, responsável por ponderar a influência das observações na atualização do estado.}
}

\newglossaryentry{cost-function}{
  name={\ensuremath{\mathcal{J}}},
  description={Função custo que mede o desvio entre o estado analisado e o background e entre as observações e suas estimativas.}
}

\newglossaryentry{h-operator}{
  name={\ensuremath{\mat{H}}},
  description={Operador de observação (ou seu linearizado), que transforma o espaço do modelo no espaço das observações.}
}

%----------------------------------------------------------------------
% TERMOS CONCEITUAIS (exemplo de glossário descritivo)
%----------------------------------------------------------------------
\newglossaryentry{analise}{
  name={Análise},
  description={Resultado do processo de assimilação de dados — um campo que combina previsões de modelo com observações, ponderadas por suas incertezas.}
}

\newglossaryentry{background}{
  name={Background},
  description={Estimativa inicial (geralmente uma previsão de curto prazo) usada como ponto de partida na assimilação.}
}
\newglossaryentry{chute-inicial}{
  name={Chute inicial},
  description={Estimativa de curto prazo do estado atmosférico utilizada como ponto de partida no processo de assimilação de dados. 
  Também chamada de \textit{first guess} ou \textit{background}, é obtida normalmente pela integração anterior do modelo numérico e representa a melhor estimativa disponível antes da incorporação das observações mais recentes.}
}
\newglossaryentry{blue}{
  name={Melhor estimativa estatística (BLUE)},
  description={Acrônimo de \textit{Best Linear Unbiased Estimator}, ou melhor estimador linear não-viesado. 
  Representa a estimativa que minimiza a variância do erro entre o estado analisado e o verdadeiro, sob as hipóteses de linearidade, gaussianidade e não-viesamento dos erros. 
  Constitui a base teórica da Interpolação Ótima e do Filtro de Kalman, formando o princípio estatístico fundamental da assimilação de dados.}
}

\newglossaryentry{forecast}{
  name={Previsão},
  description={Resultado da integração do modelo a partir da análise ou de um estado inicial fornecido.}
}

\newglossaryentry{variacional}{
  name={Método Variacional},
  description={Técnica de assimilação que encontra o estado ótimo minimizando uma função custo em um espaço contínuo.}
}

\newglossaryentry{ensemble}{
  name={Ensemble},
  description={Conjunto de múltiplas previsões do modelo com pequenas perturbações iniciais, usado para representar incertezas.}
}

%======================================================================
% FIM DO ARQUIVO
%======================================================================



%----------------------------------------------------------------------
% Informações do documento
%----------------------------------------------------------------------
\title{Introdução à Assimilação de Dados\\
\large Fundamentos, prática e perspectivas}
\author{João Gerd Zell de Mattos}
\date{\today}

%======================================================================
\begin{document}
%======================================================================

%----------------------------------------------------------------------
% Capa personalizada
%----------------------------------------------------------------------
\begin{titlepage}
  \centering
  \vspace*{1cm}
  \fcolorbox{white}{primary}{%
    \begin{minipage}{0.95\linewidth}
      \vspace*{1.6cm}
      \centering
      {\bfseries\fontsize{28pt}{30pt}\selectfont\color{white} Introdução à Assimilação de Dados\par}
      \vspace{0.6cm}
      {\Large\color{white} Fundamentos, prática e perspectivas\par}
      \vspace{1.6cm}
    \end{minipage}
  }
  \vfill
  {\Large\color{textgray} \textbf{João Gerd Zell de Mattos}\par}
  \vspace{0.25cm}
  {\large\color{textgray} \today\par}
  \vspace{1.5cm}
  \begin{center}
    \rule{0.35\linewidth}{0.6pt}\par
    \vspace{0.2cm}
    {\small\color{textgray} (rascunho / edição do autor)}
  \end{center}
\end{titlepage}

%----------------------------------------------------------------------
% Parte inicial (não numerada)
%----------------------------------------------------------------------
\frontmatter

%---------------------------- Dedicatória -----------------------------
\chapter*{Dedicatória}
\addcontentsline{toc}{chapter}{Dedicatória}
\vspace*{2cm}
\begin{flushright}
\emph{A quem busca aproximar modelos e mundo real.}
\end{flushright}

%----------------------------- Epígrafe -------------------------------
\chapter*{Epígrafe}
\addcontentsline{toc}{chapter}{Epígrafe}
\epigraph{Entre o que o modelo prevê e o que o instrumento observa, buscamos a melhor estimativa possível.}{\textit{Autor desconhecido}}

%-------------------------- Agradecimentos ----------------------------
\chapter*{Agradecimentos}
\addcontentsline{toc}{chapter}{Agradecimentos}
Agradeço a colegas, estudantes e instituições que contribuíram com dados, ideias e discussões sobre \gls{ad}.

%------------------------------ Prefácio ------------------------------
\chapter*{Prefácio}
\addcontentsline{toc}{chapter}{Prefácio}
Este livro introduz a \gls{ad} a partir de uma trajetória intuitiva: da interpolação e ajuste aos métodos modernos (VAR, EnKF, híbridos), com foco em aplicações operacionais.

% Sumário
\tableofcontents

%======================================================================
\mainmatter
%======================================================================

%======================================================================
% PARTE I — O CAMINHO ATÉ A ASSIMILAÇÃO DE DADOS
%======================================================================
\part{O caminho até a assimilação de dados}

%=========================================================
% Capítulo 1 — A arte de combinar informações
%=========================================================
\chapter{A arte de combinar informações}
\label{ch:arte-combinar}

\noindent\textbf{Resumo:}
Este capítulo apresenta a ideia fundamental da assimilação de dados (AD) como a arte de combinar diferentes fontes de informação --- o modelo numérico (background) e as observações --- para aproximar-se do estado real do sistema. Partimos da interpolação clássica (estimativas ponderadas no espaço) e damos o salto conceitual para pesos baseados em erro, preparando o terreno para a formulação estatística moderna da AD.\footnote{Para uma abordagem didática inicial de interpolação e motivação geométrica, ver \emph{Lecture 2 --- Interpolation}. :contentReference[oaicite:0]{index=0}}

%---------------------------------------------------------
\section{Entre o modelo e a realidade}
A atmosfera é contínua, mas nossas medições são pontuais e esparsas no espaço e no tempo. Termômetros, radiossondas, boias e satélites amostram apenas uma fração do estado verdadeiro. Para compreender ou prever esse estado, é preciso preencher lacunas. A \emph{interpolação} é o primeiro passo intuitivo: estimar valores em pontos não observados a partir de medições vizinhas. Em AD, esse raciocínio evolui para uma combinação \emph{ótima} entre conhecimento de modelo e dados observados, sempre levando em conta as \emph{incertezas}.

%---------------------------------------------------------
\section{A interpolação como estimativa ponderada}
Na forma mais simples, a interpolação pode ser descrita como uma combinação linear ponderada:
\begin{equation}
\hat{z}(\mathbf{x}_0) \;=\; \sum_{i=1}^{N} w_i(\mathbf{x}_0)\, z(\mathbf{x}_i),
\qquad
\sum_{i=1}^{N} w_i(\mathbf{x}_0) = 1,
\label{eq:interp-linear}
\end{equation}
em que $z(\mathbf{x}_i)$ são observações conhecidas e $w_i(\mathbf{x}_0)$ são pesos que decaem com a distância entre o ponto de interesse $\mathbf{x}_0$ e as posições $\mathbf{x}_i$. Um exemplo comum é o uso de funções de influência suaves, como uma gaussiana:
\begin{equation}
w_i(\mathbf{x}_0) = \frac{\exp\!\big(-\|\mathbf{x}_0-\mathbf{x}_i\|^2/R^2\big)}{\sum_{j=1}^{N} \exp\!\big(-\|\mathbf{x}_0-\mathbf{x}_j\|^2/R^2\big)},
\label{eq:pesos-gauss}
\end{equation}
onde $R$ representa um raio de influência. Esse tipo de ponderação contínua fundamenta a ideia de \emph{análise objetiva}, precursora histórica da AD.

%---------------------------------------------------------
\section{A transição para pesos baseados em erro}
Na AD moderna, os pesos deixam de depender apenas da distância e passam a refletir o \emph{erro estatístico} das fontes de informação. Seja $x_b$ o \emph{background} (previsão de curto prazo do modelo), $y$ o vetor de observações e $H$ o operador que mapeia o estado do modelo para o espaço de observação. A análise $x_a$ é dada por:
\begin{equation}
x_a \;=\; x_b \;+\; K\big(y - Hx_b\big),
\label{eq:analise-kalman}
\end{equation}
onde a matriz de ganho de Kalman $K$ define os \emph{pesos ótimos}:
\begin{equation}
K \;=\; B\,H^\top\big(H\,B\,H^\top + R\big)^{-1}.
\label{eq:kalman-gain}
\end{equation}
Em \eqref{eq:kalman-gain}, $B$ é a covariância de erro de background (modelo) e $R$ a covariância de erro de observação. Intuitivamente:
\begin{itemize}
  \item Se $B \gg R$, as observações recebem maior peso;
  \item Se $R \gg B$, o background prevalece.
\end{itemize}
Logo, a AD pode ser vista como uma \emph{interpolação ponderada pelos erros}: uma fusão \emph{estatisticamente} fundamentada entre modelo e dados.

%---------------------------------------------------------
\section{Exemplo escalar (BLUE 1D)}
Considere uma variável escalar $z$ observada ($z_o$) e prevista pelo modelo ($z_b$). Supondo erros não correlacionados, com variâncias $\sigma_o^2$ e $\sigma_b^2$, a melhor estimativa linear não enviesada (\emph{BLUE}) é
\begin{equation}
z_a \;=\; \frac{z_b/\sigma_b^2 + z_o/\sigma_o^2}{\,1/\sigma_b^2 + 1/\sigma_o^2\,}
\;=\;
\underbrace{\frac{1/\sigma_b^2}{1/\sigma_b^2 + 1/\sigma_o^2}}_{\text{peso do } z_b}\, z_b
\;+\;
\underbrace{\frac{1/\sigma_o^2}{1/\sigma_b^2 + 1/\sigma_o^2}}_{\text{peso do } z_o}\, z_o.
\label{eq:blue-1d}
\end{equation}
Se o modelo costuma errar mais ($\sigma_b^2$ grande), a observação domina e vice-versa. A fórmula \eqref{eq:blue-1d} é a versão 1D da ideia contida em \eqref{eq:analise-kalman}--\eqref{eq:kalman-gain}.

%---------------------------------------------------------
\section{Figuras conceituais (100\% LaTeX)}
As Figuras~\ref{fig:interp-1d} e \ref{fig:peso-distancia} são geradas com \texttt{pgfplots} e ilustram:
(i) uma interpolação 1D com pesos gaussianos; (ii) o decaimento do peso em função da distância.

\begin{figure}[h!]
\centering
\begin{tikzpicture}
  \begin{axis}[
    width=0.9\linewidth, height=6cm,
    xlabel={$x$}, ylabel={$z$},
    xmin=0, xmax=10, ymin=18, ymax=26,
    grid=both, grid style={densely dotted},
    legend style={at={(0.98,0.02)},anchor=south east, draw=none}
  ]
    % Pontos observados (ex.: temperatura)
    \addplot+[only marks, mark=*] coordinates {(2,20) (8,24)}; \addlegendentry{observações}
    % Curva de interpolação ponderada (duas gaussianas normalizadas)
    \addplot+[domain=0:10, samples=200, thick]
      { (20*exp(-(x-2)^2/6^2) + 24*exp(-(x-8)^2/6^2)) / (exp(-(x-2)^2/6^2) + exp(-(x-8)^2/6^2)) };
    \addlegendentry{$\hat{z}(x)$ com pesos gaussianos}
  \end{axis}
\end{tikzpicture}
\caption{Interpolação 1D entre duas observações usando pesos gaussianos normalizados (raio efetivo $R\approx 6$). A estimativa transita suavemente entre os valores observados.}
\label{fig:interp-1d}
\end{figure}

\begin{figure}[h!]
\centering
\begin{tikzpicture}
  \begin{axis}[
    width=0.9\linewidth, height=6cm,
    xlabel={distância $d$}, ylabel={peso $w(d)$},
    xmin=0, xmax=10, ymin=0, ymax=1.05,
    grid=both, grid style={densely dotted},
    legend style={at={(0.98,0.98)},anchor=north east, draw=none}
  ]
    \addplot+[domain=0:10, samples=200, thick]
      { exp(- (x^2) / 4^2) };
    \addlegendentry{$w(d)=\exp\!\left(-d^2/R^2\right)$, $R=4$}
  \end{axis}
\end{tikzpicture}
\caption{Função de peso gaussiana típica utilizada em análises espaciais: o peso decai rapidamente com a distância em relação ao raio $R$.}
\label{fig:peso-distancia}
\end{figure}

%---------------------------------------------------------
\section{Síntese}
A interpolação fornece a intuição: estimar nos vazios combinando vizinhança e suavidade. A AD amplia essa ideia ao incluir \emph{conhecimento de erro} --- tanto do modelo quanto das observações --- e formular a estimativa como uma \emph{combinação ótima}. Nos próximos capítulos, formalizaremos essa transição: dos ajustes (mínimos quadrados) à análise objetiva e, então, aos métodos modernos (VAR, EnKF e Kalman) que compõem o arsenal atual da assimilação de dados.

% Fim do Capítulo 1
 % Cap. 1 — A arte de combinar informações
%%=========================================================
% Capítulo 2 — Interpolação: o ponto de partida
%=========================================================
\chapter{Interpolação: o ponto de partida}
\label{ch:interpolacao}

\noindent\textbf{Resumo:}
Este capítulo introduz os fundamentos da interpolação como método de estimar valores em pontos desconhecidos a partir de dados observados. Apresenta exemplos de interpolação linear, polinomial e spline, discutindo seus limites e a motivação para evoluir em direção ao método dos mínimos quadrados. É aqui que nasce a noção de ``melhor estimativa'' em vez de ``ajuste exato''.

%---------------------------------------------------------
\section{Definição e motivação}
Interpolação é o processo de construir uma função contínua $f(x)$ que passa por um conjunto de pontos conhecidos $(x_i, y_i)$, $i = 1,2,\dots,n$.  
A forma mais simples é a \emph{interpolação linear}, que conecta os pontos por segmentos de reta:
\begin{equation}
f(x) = y_i + \frac{y_{i+1}-y_i}{x_{i+1}-x_i}\,(x-x_i), \quad x_i \le x \le x_{i+1}.
\label{eq:interp-linear}
\end{equation}
O resultado é uma aproximação por partes, contínua mas não suave nas junções.  
Quando se deseja uma função mais lisa, recorre-se à \emph{interpolação polinomial}.

%---------------------------------------------------------
\section{Interpolação polinomial}
Dado $n$ pontos, existe um polinômio de grau $n-1$ que os interpola exatamente:
\begin{equation}
P(x) = a_0 + a_1x + a_2x^2 + \dots + a_{n-1}x^{n-1}.
\label{eq:poly}
\end{equation}
Os coeficientes $a_k$ são determinados impondo $P(x_i)=y_i$.  
Por exemplo, para três pontos $(x_1,y_1)$, $(x_2,y_2)$ e $(x_3,y_3)$:
\begin{equation}
\begin{bmatrix}
1 & x_1 & x_1^2 \\
1 & x_2 & x_2^2 \\
1 & x_3 & x_3^2
\end{bmatrix}
\begin{bmatrix}
a_0 \\ a_1 \\ a_2
\end{bmatrix}
=
\begin{bmatrix}
y_1 \\ y_2 \\ y_3
\end{bmatrix}.
\label{eq:vandermonde}
\end{equation}
Essa é a \emph{matriz de Vandermonde}, cuja inversão dá a solução exata.  
Entretanto, quando o número de pontos aumenta, o polinômio tende a oscilar fortemente entre as amostras — o chamado \emph{efeito de Runge} — tornando-se inadequado para dados reais ruidosos.

%---------------------------------------------------------
\section{Interpolação por splines}
Para obter suavidade sem oscilações, utilizam-se \emph{splines}, especialmente o \emph{spline cúbico}, definido por polinômios de grau 3 em cada subintervalo $[x_i, x_{i+1}]$, com continuidade nas primeiras e segundas derivadas:
\[
S_i(x) = a_i + b_i(x-x_i) + c_i(x-x_i)^2 + d_i(x-x_i)^3.
\]
O spline cúbico oferece uma transição suave entre os pontos e é amplamente usado em aplicações meteorológicas, como interpolação de perfis verticais de temperatura e umidade.

%---------------------------------------------------------
\section{Limitações e surgimento do ajuste ótimo}
Na prática, medições contêm ruído e inconsistências. Se exigirmos que a função passe exatamente por todos os pontos, amplificamos os erros.  
O ideal é encontrar uma função que \emph{aproxime} os dados minimizando o erro global:
\begin{equation}
J(a_0,\dots,a_m) = \sum_{i=1}^{n} \big[y_i - f(x_i; a_0,\dots,a_m)\big]^2.
\label{eq:min-squares-cost}
\end{equation}
Essa função custo $J$ leva ao método dos \emph{mínimos quadrados}, que será formalmente desenvolvido no próximo capítulo. A interpolação, portanto, evolui naturalmente para o ajuste ótimo — uma transição conceitual essencial para a assimilação de dados, em que buscamos minimizar os erros entre o modelo e as observações.

%---------------------------------------------------------
\section{Figuras conceituais (pgfplots)}
As Figuras~\ref{fig:interp-linear} e \ref{fig:interp-poly} ilustram comparativamente a interpolação linear e polinomial entre cinco pontos.

\begin{figure}[h!]
\centering
\begin{tikzpicture}
\begin{axis}[
  width=0.9\linewidth, height=6cm,
  xlabel={$x$}, ylabel={$y$},
  xmin=0, xmax=10, ymin=0, ymax=10,
  grid=both, grid style={densely dotted},
  legend style={at={(0.98,0.02)},anchor=south east,draw=none}
]
  % dados
  \addplot+[only marks,mark=*] coordinates {(0,0) (2,2) (4,3) (6,5) (8,8) (10,9)}; \addlegendentry{dados}
  % linear
  \addplot+[const plot, thick, mark=none] coordinates {(0,0) (2,2) (4,3) (6,5) (8,8) (10,9)};
  \addlegendentry{interpolação linear}
\end{axis}
\end{tikzpicture}
\caption{Interpolação linear por segmentos de reta: simples e contínua, mas não suave nas junções.}
\label{fig:interp-linear}
\end{figure}

\begin{figure}[h!]
\centering
\begin{tikzpicture}
\begin{axis}[
  width=0.9\linewidth, height=6cm,
  xlabel={$x$}, ylabel={$y$},
  xmin=0, xmax=10, ymin=0, ymax=10,
  grid=both, grid style={densely dotted},
  legend style={at={(0.98,0.02)},anchor=south east,draw=none}
]
  \addplot+[only marks,mark=*] coordinates {(0,0) (2,2) (4,3) (6,5) (8,8) (10,9)};
  \addplot+[domain=0:10, samples=200, thick] {0.0007*x^4 - 0.016*x^3 + 0.19*x^2 + 0.37*x + 0.1};
  \addlegendentry{polinômio interpolador}
\end{axis}
\end{tikzpicture}
\caption{Interpolação polinomial de grau elevado: passa por todos os pontos, mas pode oscilar entre eles.}
\label{fig:interp-poly}
\end{figure}

%---------------------------------------------------------
\section{Síntese}
A interpolação é o ponto de partida natural da assimilação de dados: ambas buscam reconstruir um campo desconhecido a partir de informações discretas. Contudo, enquanto a interpolação tradicional busca o ajuste exato, a assimilação valoriza o equilíbrio entre coerência e confiança --- um ajuste \emph{ótimo} em vez de \emph{exato}. O passo seguinte é formalizar essa ideia através do método dos mínimos quadrados, que constitui o elo matemático entre interpolação e análise objetiva.

% Fim do Capítulo 2
 % Cap. 2 — Interpolação: o ponto de partida
%%=========================================================
% Capítulo 3 — Ajuste por mínimos quadrados
%=========================================================
\chapter{Ajuste por mínimos quadrados}
\label{ch:minimos-quadrados}

\noindent\textbf{Resumo:}
Este capítulo introduz o método dos mínimos quadrados como a extensão natural da interpolação quando os dados contêm ruído e inconsistências.  
Mostra-se como o ajuste ótimo busca minimizar o erro global e não o erro pontual, formalizando o conceito de \emph{melhor estimativa}.  
A formulação matricial do problema é apresentada, juntamente com a interpretação geométrica de projeção ortogonal, que mais tarde dará origem à formulação variacional da assimilação de dados.

%---------------------------------------------------------
\section{Motivação: do ajuste exato ao ótimo}
Em situações reais, as observações não são exatas. Cada ponto medido $y_i$ possui um erro associado, e forçar a função a passar exatamente por todos eles pode introduzir oscilações e exagerar o ruído.  
Em vez disso, buscamos uma função $f(x; \mathbf{a})$ parametrizada (por exemplo, um polinômio) que minimize o erro médio ao longo de todos os pontos:
\begin{equation}
J(\mathbf{a}) = \sum_{i=1}^{n} \big[y_i - f(x_i; \mathbf{a})\big]^2.
\label{eq:J_scalar}
\end{equation}
O valor mínimo de $J$ fornece o \emph{melhor ajuste} global — o que explica a sigla \emph{BLUE} (\emph{Best Linear Unbiased Estimator}) que aparecerá mais adiante em AD.

%---------------------------------------------------------
\section{Formulação linear e derivação}
Para um modelo linear nos parâmetros,
\begin{equation}
f(x_i;\mathbf{a}) = \sum_{j=1}^{m} a_j \, \phi_j(x_i),
\label{eq:modelo-linear}
\end{equation}
em que $\phi_j(x)$ são funções base conhecidas (por exemplo, $1$, $x$, $x^2$, ...), podemos reescrever o problema em forma matricial:
\[
\mathbf{y} = \Phi \mathbf{a} + \boldsymbol{\varepsilon},
\]
onde
\[
\Phi =
\begin{bmatrix}
\phi_1(x_1) & \phi_2(x_1) & \dots & \phi_m(x_1) \\
\phi_1(x_2) & \phi_2(x_2) & \dots & \phi_m(x_2) \\
\vdots & \vdots & \ddots & \vdots \\
\phi_1(x_n) & \phi_2(x_n) & \dots & \phi_m(x_n)
\end{bmatrix},
\qquad
\mathbf{y} = 
\begin{bmatrix}
y_1 \\ y_2 \\ \vdots \\ y_n
\end{bmatrix},
\quad
\mathbf{a} = 
\begin{bmatrix}
a_1 \\ a_2 \\ \vdots \\ a_m
\end{bmatrix}.
\]
O vetor de resíduos é $\boldsymbol{r} = \mathbf{y} - \Phi\mathbf{a}$.  
Minimizar $J=\boldsymbol{r}^\top\boldsymbol{r}$ leva à condição normal:
\begin{equation}
\frac{\partial J}{\partial \mathbf{a}} = -2 \Phi^\top (\mathbf{y} - \Phi\mathbf{a}) = 0,
\label{eq:gradJ}
\end{equation}
resultando nas \emph{equações normais}:
\begin{equation}
\Phi^\top \Phi \, \mathbf{a} = \Phi^\top \mathbf{y}.
\label{eq:normal-equations}
\end{equation}
A solução é dada por:
\begin{equation}
\boxed{\mathbf{a} = (\Phi^\top \Phi)^{-1}\Phi^\top \mathbf{y}.}
\label{eq:LS-solution}
\end{equation}
Essa expressão é a base da estimação linear: a mesma estrutura reaparece em assimilação de dados sob a forma de análise estatística ponderada.

%---------------------------------------------------------
\section{Interpretação geométrica}
A solução dos mínimos quadrados pode ser vista como uma \emph{projeção ortogonal} do vetor de observações $\mathbf{y}$ sobre o subespaço gerado pelas colunas de $\Phi$ (Figura~\ref{fig:geom-ls}).  
O vetor ajustado $\hat{\mathbf{y}} = \Phi\mathbf{a}$ é a projeção de $\mathbf{y}$ nesse subespaço, e o resíduo $\boldsymbol{r} = \mathbf{y}-\hat{\mathbf{y}}$ é ortogonal a ele:
\[
\Phi^\top \boldsymbol{r} = 0.
\]
Essa propriedade geométrica (resíduo ortogonal) é o coração da AD variacional: o mínimo da função custo ocorre quando o erro projetado é ortogonal ao espaço de sensibilidade do modelo.

\begin{figure}[h!]
\centering
\begin{tikzpicture}
  \begin{axis}[
    width=0.75\linewidth, height=7cm,
    view={30}{30},
    axis lines=center,
    xlabel={$a_1\phi_1$}, ylabel={$a_2\phi_2$}, zlabel={$y$},
    xtick=\empty, ytick=\empty, ztick=\empty,
    grid=major, grid style={densely dotted}
  ]
    % plano Phi*a
    \addplot3[surf, opacity=0.15, samples=15, domain=0:1, y domain=0:1]
      {2*x + 3*y};
    % vetor y
    \addplot3+[->, thick] coordinates {(0.4,0.2,0) (0.4,0.2,3.6)};
    % vetor projeção
    \addplot3+[->, thick, color=brandB] coordinates {(0.4,0.2,0) (0.4,0.2,1.6)};
    % vetor resíduo
    \addplot3+[->, thick, color=brandA] coordinates {(0.4,0.2,1.6) (0.4,0.2,3.6)};
    \node[anchor=west] at (axis cs:0.5,0.3,3.5) {$\mathbf{y}$};
    \node[anchor=west, color=brandB] at (axis cs:0.5,0.3,1.6) {$\hat{\mathbf{y}}=\Phi\mathbf{a}$};
    \node[anchor=west, color=brandA] at (axis cs:0.5,0.3,2.5) {$\boldsymbol{r}$};
  \end{axis}
\end{tikzpicture}
\caption{Interpretação geométrica dos mínimos quadrados: $\mathbf{y}$ é projetado no subespaço gerado por $\Phi$. O resíduo $\boldsymbol{r}$ é ortogonal ao plano.}
\label{fig:geom-ls}
\end{figure}

%---------------------------------------------------------
\section{Exemplo numérico simples}
Considere o ajuste de uma reta $y = a_0 + a_1x$ a três pontos: $(0,1)$, $(1,2)$ e $(2,2)$.  
Temos:
\[
\Phi = 
\begin{bmatrix}
1 & 0 \\
1 & 1 \\
1 & 2
\end{bmatrix}, \quad
\mathbf{y} =
\begin{bmatrix}
1 \\ 2 \\ 2
\end{bmatrix}.
\]
Aplicando \eqref{eq:LS-solution}:
\[
\Phi^\top\Phi =
\begin{bmatrix}
3 & 3 \\ 3 & 5
\end{bmatrix},
\qquad
\Phi^\top\mathbf{y} =
\begin{bmatrix}
5 \\ 6
\end{bmatrix},
\]
resultando em
\[
\mathbf{a} =
\begin{bmatrix}
1.0 \\ 0.5
\end{bmatrix}.
\]
Logo, a reta ajustada é $y = 1 + 0.5x$.  
O erro global é minimizado sem exigir que o modelo passe exatamente por todos os pontos (Figura~\ref{fig:ls-fit}).

\begin{figure}[h!]
\centering
\begin{tikzpicture}
\begin{axis}[
  width=0.9\linewidth, height=6cm,
  xlabel={$x$}, ylabel={$y$},
  xmin=-0.5, xmax=2.5, ymin=0, ymax=3,
  grid=both, grid style={densely dotted},
  legend style={at={(0.02,0.98)},anchor=north west,draw=none}
]
  \addplot+[only marks,mark=*] coordinates {(0,1) (1,2) (2,2)}; \addlegendentry{dados}
  \addplot+[domain=-0.5:2.5, samples=200, thick] {1 + 0.5*x}; \addlegendentry{ajuste $y=1+0.5x$}
\end{axis}
\end{tikzpicture}
\caption{Exemplo de ajuste linear por mínimos quadrados. A reta não passa por todos os pontos, mas minimiza o erro global.}
\label{fig:ls-fit}
\end{figure}

%---------------------------------------------------------
\section{Ligação com assimilação de dados}
O método dos mínimos quadrados é a espinha dorsal da formulação estatística da assimilação de dados.  
Na AD, o funcional a minimizar tem estrutura análoga à de \eqref{eq:J_scalar}, porém com dois termos principais:
\begin{equation}
J(x) = (x - x_b)^\top B^{-1} (x - x_b)
     + (y - Hx)^\top R^{-1} (y - Hx),
\label{eq:J-DA}
\end{equation}
em que o primeiro termo mede o erro do modelo e o segundo, o das observações.  
Minimizar $J$ equivale a resolver uma versão generalizada do problema de mínimos quadrados ponderado por covariâncias — uma generalização direta de \eqref{eq:normal-equations}.

%---------------------------------------------------------
\section{Síntese}
O método dos mínimos quadrados marca a transição entre a interpolação determinística e a assimilação estatística.  
O conceito de \emph{melhor estimativa global} substitui o de \emph{ajuste exato}, preparando o caminho para a \emph{análise objetiva} e, mais adiante, para a formulação completa da assimilação de dados.  
No próximo capítulo, exploraremos como essa ideia se expande para duas dimensões e como a suavização controla o equilíbrio entre fidelidade e coerência espacial.

% Fim do Capítulo 3
 % Cap. 3 — Ajuste por mínimos quadrados
%%=========================================================
% Capítulo 4 — Ajustes em duas dimensões e função de suavização
%=========================================================
\chapter{Ajustes em duas dimensões e função de suavização}
\label{ch:ajustes2d}

\noindent\textbf{Resumo:}
Neste capítulo, generalizamos o método dos mínimos quadrados para duas dimensões e introduzimos o conceito de \emph{suavização}.  
Mostra-se como o controle da variabilidade espacial conduz naturalmente aos métodos de análise objetiva usados em meteorologia e assimilação de dados.  
A ideia central é encontrar o equilíbrio entre fidelidade aos dados e coerência espacial do campo analisado.

%---------------------------------------------------------
\section{Extensão para duas dimensões}
Até agora tratamos de variáveis unidimensionais.  
Em meteorologia, porém, buscamos reconstruir campos como temperatura, pressão ou vento sobre uma área bidimensional.  
Dado um conjunto de observações $\{(x_i, y_i, z_i)\}_{i=1}^{N}$, queremos estimar uma função contínua $f(x,y)$ que represente o campo.  
Podemos expressá-la em uma base bidimensional:
\begin{equation}
f(x,y) = \sum_{j=1}^{M} a_j \, \phi_j(x,y),
\label{eq:basis2d}
\end{equation}
onde $\phi_j$ são funções de base (por exemplo, polinômios em $x$ e $y$ ou funções radiais).  
A minimização do erro global
\begin{equation}
J = \sum_{i=1}^{N} \big[z_i - f(x_i,y_i)\big]^2
\label{eq:2d-cost}
\end{equation}
segue a mesma lógica do capítulo anterior, resultando em equações normais análogas.  
A diferença está no tipo de função base escolhida e na necessidade de regularizar o ajuste.

%---------------------------------------------------------
\section{A necessidade de suavização}
Quando o número de pontos é grande ou o campo contém ruído, ajustes exatos produzem oscilações não físicas (Figura~\ref{fig:2d-raw}).  
Introduzimos então um termo de \emph{suavização} na função custo:
\begin{equation}
J_s = \sum_{i=1}^{N} \big[z_i - f(x_i,y_i)\big]^2
+ \lambda \iint \left[ \left(\frac{\partial^2 f}{\partial x^2}\right)^2
+ \left(\frac{\partial^2 f}{\partial y^2}\right)^2 \right] dx\,dy.
\label{eq:smooth-cost}
\end{equation}
O parâmetro $\lambda$ controla o compromisso entre fidelidade e suavidade:
\begin{itemize}
  \item $\lambda \to 0$: ajuste quase exato (pode oscilar);
  \item $\lambda \to \infty$: campo excessivamente suave (subestima variações reais).
\end{itemize}
Essa regularização dá origem às técnicas de \emph{análise objetiva}, como as de Cressman e Barnes, nas quais o campo é filtrado por uma função de ponderação espacial.

%---------------------------------------------------------
\section{Interpretação física e estatística}
Em termos físicos, o termo de suavização impõe coerência espacial, equivalente a exigir que o campo tenha continuidade e derivadas limitadas — algo análogo à difusão.  
Em termos estatísticos, a regularização equivale a assumir correlações espaciais finitas entre os erros.  
Na assimilação de dados, essa correlação é descrita pela matriz de covariância $B$, cuja estrutura (gaussiana, exponencial, etc.) define a “suavidade” da análise.

%---------------------------------------------------------
\section{Funções de ponderação e filtros espaciais}
A interpolação bidimensional pode ser escrita como:
\begin{equation}
\hat{f}(x,y) = \frac{\displaystyle \sum_{i=1}^{N} W_i(x,y)\, z_i}
                     {\displaystyle \sum_{i=1}^{N} W_i(x,y)},
\label{eq:weighted2d}
\end{equation}
em que $W_i(x,y)$ são funções de peso que decaem com a distância:
\[
W_i(x,y) = \exp\!\left[-\frac{(x-x_i)^2 + (y-y_i)^2}{R^2}\right],
\]
com $R$ sendo o raio de influência.  
Essa forma é o ponto de partida para o método de Barnes e outros esquemas de análise objetiva, em que $R$ e $\lambda$ são ajustados para balancear ruído e detalhe.

%---------------------------------------------------------
\section{Exemplo visual em 2D (pgfplots)}
As Figuras~\ref{fig:2d-raw} e \ref{fig:2d-smooth} ilustram o efeito da suavização sobre um campo de pontos amostrados aleatoriamente.

\begin{figure}[h!]
\centering
\begin{tikzpicture}
  \begin{axis}[
    view={30}{40},
    width=0.8\linewidth, height=7cm,
    xlabel={$x$}, ylabel={$y$}, zlabel={$z$},
    domain=0:4, y domain=0:4,
    grid=both, samples=15,
    colormap/viridis,
  ]
    \addplot3[surf, opacity=0.8, shader=interp]
      {sin(deg(x*1.2)) + cos(deg(y*1.3)) + 0.3*rnd};
  \end{axis}
\end{tikzpicture}
\caption{Campo interpolado diretamente (sem suavização): as flutuações aleatórias geram ruído visível.}
\label{fig:2d-raw}
\end{figure}

\begin{figure}[h!]
\centering
\begin{tikzpicture}
  \begin{axis}[
    view={30}{40},
    width=0.8\linewidth, height=7cm,
    xlabel={$x$}, ylabel={$y$}, zlabel={$z$},
    domain=0:4, y domain=0:4,
    grid=both, samples=15,
    colormap/viridis,
  ]
    \addplot3[surf, opacity=0.8, shader=interp]
      {sin(deg(x*1.2)) + cos(deg(y*1.3))};
  \end{axis}
\end{tikzpicture}
\caption{Campo suavizado com regularização: o ruído é atenuado, preservando as estruturas principais.}
\label{fig:2d-smooth}
\end{figure}

%---------------------------------------------------------
\section{Conexão com análise objetiva}
A adição do termo de suavização em \eqref{eq:smooth-cost} e o uso de pesos espaciais em \eqref{eq:weighted2d} constituem a base da \emph{análise objetiva} (\emph{Objective Analysis} ou OBAN).  
Essas ideias foram aplicadas na meteorologia operacional antes da era dos supercomputadores e ainda hoje fundamentam métodos como Cressman (1959) e Barnes (1964).  
A diferença é que, na assimilação moderna, a suavização e os pesos são determinados de forma \emph{ótima} a partir das estatísticas de erro — um avanço direto do que começou aqui.

%---------------------------------------------------------
\section{Síntese}
A generalização para duas dimensões e a introdução da suavização marcaram a transição da interpolação geométrica para a análise estatística.  
Os parâmetros de suavidade e raio de influência representam fisicamente a correlação espacial dos erros e matematicamente o filtro aplicado ao campo analisado.  
No próximo capítulo, exploraremos como esses conceitos foram formalizados nos métodos clássicos de análise objetiva — a ponte definitiva entre interpolação e assimilação de dados.

% Fim do Capítulo 4
 % Cap. 4 — Ajustes em 2D e suavização
%%=========================================================
% Capítulo 5 — Análise objetiva e o nascimento da assimilação de dados
%=========================================================
\chapter{Análise objetiva e o nascimento da assimilação de dados}
\label{ch:obana}

\noindent\textbf{Resumo:}
Este capítulo apresenta a transição da interpolação clássica para os métodos de \emph{análise objetiva} (OBAN), que introduziram sistematicamente a ponderação por distância e a suavização espacial no tratamento de dados meteorológicos.  
Exploramos os esquemas de Cressman e Barnes, o método de correção sucessiva e a ideia de \emph{resposta do filtro}.  
Esses conceitos constituem o elo histórico entre as técnicas empíricas e a formulação estatística moderna da assimilação de dados.

%---------------------------------------------------------
\section{O contexto histórico}
Antes da era dos computadores de alta performance, os meteorologistas precisavam transformar medições pontuais em campos contínuos para análise sinótica.  
Esse processo — traçar isolinhas de pressão, temperatura e vento a partir de observações dispersas — era feito manualmente.  
A necessidade de automatizar e padronizar esse procedimento levou ao desenvolvimento da \emph{análise objetiva} (\emph{Objective Analysis}, OBAN).  
A OBAN formalizou a ideia de interpolar com base em \emph{funções de influência espacial} e \emph{iterações sucessivas de correção}, um precursor direto das equações da assimilação de dados.

%---------------------------------------------------------
\section{O método de Cressman (1959)}
O método de Cressman foi um dos primeiros esquemas objetivos amplamente utilizados.  
A ideia básica é estimar o campo em um ponto de grade $P$ a partir das observações $O_i$ dentro de um raio de influência $R$:
\begin{equation}
A(P) = \frac{\displaystyle \sum_i W_i \big(O_i - B_i\big)}{\displaystyle \sum_i W_i},
\qquad
W_i = \frac{R^2 - d_i^2}{R^2 + d_i^2},
\label{eq:cressman}
\end{equation}
onde $B_i$ é o valor de background (ou estimativa inicial) no ponto da observação e $d_i$ é a distância entre $P$ e $O_i$.  
Após calcular as correções $A(P)$, o campo analisado é obtido pela atualização:
\begin{equation}
B'(P) = B(P) + A(P).
\label{eq:update}
\end{equation}
O processo é repetido em várias iterações, diminuindo $R$ a cada passo, o que suaviza o campo gradualmente e refina detalhes locais.

%---------------------------------------------------------
\section{O método de Barnes (1964)}
Barnes propôs um método mais sofisticado, baseado em ponderações gaussianas, que garantem suavidade e convergência mais controlada:
\begin{equation}
A(P) = \frac{\displaystyle \sum_i \exp\!\left[-\frac{d_i^2}{\kappa}\right] (O_i - B_i)}{\displaystyle \sum_i \exp\!\left[-\frac{d_i^2}{\kappa}\right]},
\label{eq:barnes}
\end{equation}
em que $\kappa$ é um parâmetro de espalhamento que controla o raio efetivo de influência.  
O método de Barnes pode ser interpretado como uma convolução do campo de resíduos com uma função gaussiana — uma operação análoga a um \emph{filtro passa-baixa} espacial.

Barnes também introduziu a noção de \emph{resposta do filtro}, ou seja, como diferentes escalas espaciais do sinal (estrutura real) e do ruído (oscilações espúrias) são atenuadas.  
Essa noção de resposta espectral se tornaria crucial para o entendimento dos métodos de assimilação e dos filtros de Kalman.

%---------------------------------------------------------
\section{Correção sucessiva}
Em muitos esquemas, as correções $A(P)$ são aplicadas iterativamente:
\begin{equation}
B_{k+1}(P) = B_k(P) + \alpha_k A_k(P),
\label{eq:successive}
\end{equation}
onde $\alpha_k$ é um fator de relaxação (tipicamente entre 0.3 e 0.7).  
Cada iteração reduz o erro residual e melhora a consistência entre o campo de fundo e as observações.  
Esse processo lembra fortemente a estrutura iterativa dos esquemas variacionais e do filtro de Kalman — em que o campo é atualizado progressivamente à medida que novas observações são assimiladas.

%---------------------------------------------------------
\section{Resposta do filtro}
O conceito de resposta do filtro mede quanto da amplitude de uma onda espacial de determinado comprimento $\lambda$ é preservada pela análise.  
Se a resposta é $G(k)$, onde $k=2\pi/\lambda$, temos:
\[
G(k) = \frac{\text{Amplitude da análise}}{\text{Amplitude do sinal real}}.
\]
Idealmente, $G(k)\approx1$ para escalas grandes (fenômenos reais) e $G(k)\to0$ para escalas pequenas (ruído).  
Na prática, os parâmetros $R$, $\kappa$ e $\lambda$ determinam o formato de $G(k)$ e, portanto, a suavidade da análise.

A Figura~\ref{fig:filter-response} ilustra uma resposta típica de filtro para diferentes valores de $\kappa$.

\begin{figure}[h!]
\centering
\begin{tikzpicture}
\begin{axis}[
  width=0.9\linewidth, height=6cm,
  xlabel={número de onda $k$}, ylabel={resposta $G(k)$},
  xmin=0, xmax=3, ymin=0, ymax=1.05,
  grid=both, grid style={densely dotted},
  legend style={at={(0.98,0.98)},anchor=north east,draw=none}
]
  \addplot+[domain=0:3, samples=200, thick] {exp(-x^2/0.3)}; \addlegendentry{$\kappa=0.3$ (forte suavização)}
  \addplot+[domain=0:3, samples=200, thick, dashed] {exp(-x^2/1)}; \addlegendentry{$\kappa=1.0$ (moderada)}
  \addplot+[domain=0:3, samples=200, thick, dotted] {exp(-x^2/3)}; \addlegendentry{$\kappa=3.0$ (leve)}
\end{axis}
\end{tikzpicture}
\caption{Resposta típica de filtro espacial no método de Barnes: quanto menor $\kappa$, maior a atenuação de pequenas escalas.}
\label{fig:filter-response}
\end{figure}

%---------------------------------------------------------
\section{Da análise objetiva à assimilação de dados}
A estrutura matemática de \eqref{eq:cressman}--\eqref{eq:successive} já contém os elementos da assimilação moderna:
\begin{itemize}
  \item uma estimativa de \emph{background} ($B$);
  \item um campo de \emph{resíduos observacionais} ($O_i - B_i$);
  \item pesos dependentes de distância (análogos às covariâncias espaciais);
  \item atualizações iterativas (análogas ao ganho de Kalman).
\end{itemize}
Com o avanço da capacidade computacional, essas ideias evoluíram para métodos estatísticos completos, onde os pesos não são mais empíricos, mas derivados das matrizes de covariância de erro.

%---------------------------------------------------------
\section{Síntese}
A análise objetiva representou o primeiro passo concreto rumo à assimilação de dados.  
Ela trouxe à meteorologia o conceito de ponderação espacial, correção iterativa e filtragem de ruído — todos os fundamentos do pensamento estatístico posterior.  
Do traçado manual de isolinhas, a OBAN levou à base teórica que permitiria, décadas depois, a formulação completa do filtro de Kalman e dos métodos VAR.  
No próximo capítulo, introduziremos formalmente a \emph{formulação estatística da assimilação de dados}, onde o peso de cada informação é definido de maneira ótima a partir dos erros.

% Fim do Capítulo 5
 % Cap. 5 — Análise objetiva (Cressman, Barnes)
%
%%======================================================================
%% PARTE II — A ASSIMILAÇÃO DE DADOS MODERNA
%%======================================================================
%\part{A assimilação de dados moderna}
%
%%=========================================================
% Capítulo 6 — Formulação estatística da assimilação de dados
%=========================================================
\chapter{Formulação estatística da assimilação de dados}
\label{ch:formulacao-estatistica}

\noindent\textbf{Resumo:}
Neste capítulo, apresentamos a base estatística da assimilação de dados.  
A assimilação é vista como uma estimativa ótima que combina informações de um modelo (background) e observações (dados), ponderadas pelos respectivos erros.  
Mostra-se que o problema é equivalente ao dos mínimos quadrados ponderados, conduzindo à equação clássica da análise e ao ganho de Kalman.

%---------------------------------------------------------
\section{Motivação estatística}
Toda medição contém incerteza.  
Do mesmo modo, todo modelo numérico é uma representação aproximada da realidade.  
A assimilação de dados reconhece explicitamente essas imperfeições e procura combinar as fontes de informação de modo a minimizar o erro médio quadrático.

Denotemos:
\begin{itemize}
  \item $\mathbf{x}_b$ — vetor de \textbf{background} (ou previsão de curto prazo);
  \item $\mathbf{y}$ — vetor de \textbf{observações};
  \item $\mathbf{H}$ — operador que transforma o estado do modelo em espaço de observação;
  \item $\mathbf{x}_a$ — vetor \textbf{analisado} (resultado da assimilação).
\end{itemize}

O problema é então formular uma estimativa $\mathbf{x}_a$ que combine coerentemente as duas fontes:
\begin{equation}
\mathbf{x}_a = \mathbf{x}_b + \mathbf{K}\,(\mathbf{y} - \mathbf{H}\mathbf{x}_b),
\label{eq:analysis}
\end{equation}
onde $\mathbf{K}$ é a \emph{matriz de ganho} (ou matriz de ponderação ótima).

%---------------------------------------------------------
\section{Matriz de covariância e erro}
Seja o erro de background $\boldsymbol{\epsilon}_b = \mathbf{x}_b - \mathbf{x}_t$, onde $\mathbf{x}_t$ é o estado verdadeiro desconhecido.  
A matriz de covariância associada é:
\[
\mathbf{B} = \mathrm{E}[\boldsymbol{\epsilon}_b \boldsymbol{\epsilon}_b^\top].
\]
De forma análoga, o erro de observação $\boldsymbol{\epsilon}_o = \mathbf{y} - \mathbf{H}\mathbf{x}_t$ possui covariância:
\[
\mathbf{R} = \mathrm{E}[\boldsymbol{\epsilon}_o \boldsymbol{\epsilon}_o^\top].
\]
Essas matrizes representam o \emph{grau de confiança} nas fontes de informação.  
Valores pequenos em $\mathbf{R}$ indicam observações precisas; valores grandes em $\mathbf{B}$ indicam grande incerteza no modelo.

%---------------------------------------------------------
\section{Função custo e mínimos quadrados ponderados}
A análise é obtida minimizando a função custo:
\begin{equation}
J(\mathbf{x}) = (\mathbf{x} - \mathbf{x}_b)^\top \mathbf{B}^{-1} (\mathbf{x} - \mathbf{x}_b)
+ (\mathbf{y} - \mathbf{H}\mathbf{x})^\top \mathbf{R}^{-1} (\mathbf{y} - \mathbf{H}\mathbf{x}).
\label{eq:Jstat}
\end{equation}
O primeiro termo mede o desvio em relação ao modelo, ponderado por $\mathbf{B}^{-1}$, e o segundo, o desvio em relação às observações, ponderado por $\mathbf{R}^{-1}$.  
A minimização de \eqref{eq:Jstat} é análoga ao problema de mínimos quadrados com pesos.

Derivando $J$ em relação a $\mathbf{x}$ e igualando a zero:
\[
\frac{\partial J}{\partial \mathbf{x}} = 2\mathbf{B}^{-1}(\mathbf{x}-\mathbf{x}_b)
- 2\mathbf{H}^\top \mathbf{R}^{-1}(\mathbf{y}-\mathbf{H}\mathbf{x}) = 0,
\]
obtemos:
\[
(\mathbf{B}^{-1} + \mathbf{H}^\top \mathbf{R}^{-1}\mathbf{H})\mathbf{x}_a
= \mathbf{B}^{-1}\mathbf{x}_b + \mathbf{H}^\top \mathbf{R}^{-1}\mathbf{y}.
\]
A solução, após manipulação algébrica, conduz à forma clássica:
\begin{equation}
\boxed{\mathbf{x}_a = \mathbf{x}_b + \mathbf{K}(\mathbf{y} - \mathbf{H}\mathbf{x}_b),}
\label{eq:xa-final}
\end{equation}
onde
\begin{equation}
\boxed{\mathbf{K} = \mathbf{B}\mathbf{H}^\top (\mathbf{H}\mathbf{B}\mathbf{H}^\top + \mathbf{R})^{-1}.}
\label{eq:kalman-gain}
\end{equation}

%---------------------------------------------------------
\section{Interpretação do ganho de Kalman}
A matriz $\mathbf{K}$ define quanto da inovação $(\mathbf{y}-\mathbf{H}\mathbf{x}_b)$ deve ser aplicada para corrigir o background.  
Se as observações são muito confiáveis ($\mathbf{R}$ pequeno), $\mathbf{K}$ se aproxima de $\mathbf{H}^{-1}$, e a análise tende às observações.  
Se o modelo é mais confiável ($\mathbf{B}$ pequeno), $\mathbf{K}$ tende a zero, e a análise permanece próxima do background.

A equação \eqref{eq:kalman-gain} mostra explicitamente a ponderação por erro — o mesmo princípio intuitivo que vimos no exemplo unidimensional no Capítulo~\ref{ch:interpolacao}:
\[
T_a = \frac{T_m/\sigma_m^2 + T_o/\sigma_o^2}{1/\sigma_m^2 + 1/\sigma_o^2}.
\]

%---------------------------------------------------------
\section{Resposta do filtro e propagação de erro}
O ganho $\mathbf{K}$ atua como um filtro espacial e estatístico.  
Aplicado a uma sequência de atualizações, ele controla quanto das estruturas observadas é incorporado e quanto do ruído é atenuado.  
A covariância do erro de análise é:
\begin{equation}
\mathbf{A} = (\mathbf{I} - \mathbf{K}\mathbf{H})\,\mathbf{B}.
\label{eq:cov-analysis}
\end{equation}
Essa expressão mostra que o erro de análise é sempre menor (ou igual) ao erro de background, refletindo a melhoria obtida pela assimilação.

\begin{figure}[h!]
\centering
\begin{tikzpicture}
\begin{axis}[
  width=0.9\linewidth, height=6cm,
  xlabel={Número de onda $k$}, ylabel={Resposta $G(k)$},
  xmin=0, xmax=3, ymin=0, ymax=1.05,
  grid=both, grid style={densely dotted},
  legend style={at={(0.98,0.98)},anchor=north east,draw=none}
]
  \addplot+[domain=0:3, samples=200, thick] {1/(1 + x^2)}; \addlegendentry{filtro suave ($B>R$)}
  \addplot+[domain=0:3, samples=200, thick, dashed] {1/(1 + 0.2*x^2)}; \addlegendentry{observações dominantes ($R<B$)}
\end{axis}
\end{tikzpicture}
\caption{Resposta espectral do filtro de assimilação: o ganho $\mathbf{K}$ atua como um filtro passa-baixa, ponderando a contribuição das observações conforme a escala espacial.}
\label{fig:kalman-filter-response}
\end{figure}

%---------------------------------------------------------
\section{Síntese e ligação com os métodos anteriores}
O formalismo apresentado neste capítulo demonstra que a assimilação de dados é uma generalização estatística dos métodos de análise objetiva.  
Os pesos empíricos de Cressman e Barnes são substituídos por ponderações ótimas derivadas das matrizes $\mathbf{B}$ e $\mathbf{R}$.  
A iteração sucessiva dá lugar à atualização analítica por meio da matriz de ganho.  
Em essência, a assimilação moderna é uma \emph{interpolação ótima ponderada pelos erros estatísticos}, uma fusão entre física, estatística e computação.

No próximo capítulo, veremos como esse formalismo pode ser aplicado de forma dinâmica — com a atualização contínua do estado e da covariância — culminando no \emph{Filtro de Kalman} e em suas extensões para sistemas não lineares.

% Fim do Capítulo 6
 % Cap. 6 — Formulação estatística (B, R, K)
%%=========================================================
% Capítulo 7 — O Filtro de Kalman e suas extensões
%=========================================================
\chapter{O Filtro de Kalman e suas extensões}
\label{ch:kalman}

\noindent\textbf{Resumo:}
Este capítulo apresenta a formulação recursiva do Filtro de Kalman e suas variantes.  
O filtro de Kalman é a forma dinâmica da assimilação de dados, permitindo a atualização contínua do estado e de sua incerteza à medida que novas observações chegam.  
Discutem-se também as versões não lineares — EKF (Estendido) e EnKF (Ensemble) — aplicadas à meteorologia moderna.

%---------------------------------------------------------
\section{Do ajuste estático à estimativa dinâmica}
Nos capítulos anteriores, tratamos da análise como um problema estático:  
combinar background e observações em um único instante.  
O Filtro de Kalman generaliza essa ideia para sistemas que evoluem no tempo.

O estado do sistema $\mathbf{x}_k$ é previsto a partir do estado anterior $\mathbf{x}_{k-1}$ pelo modelo:
\begin{equation}
\mathbf{x}_k = \mathbf{M}_{k-1}\mathbf{x}_{k-1} + \boldsymbol{\eta}_{k-1},
\label{eq:model-kalman}
\end{equation}
onde $\mathbf{M}_{k-1}$ é o operador de previsão e $\boldsymbol{\eta}_{k-1}$ o erro de modelo, com covariância $\mathbf{Q}_{k-1}$.

As observações no tempo $k$ são:
\begin{equation}
\mathbf{y}_k = \mathbf{H}_k \mathbf{x}_k + \boldsymbol{\epsilon}_k,
\label{eq:obs-kalman}
\end{equation}
onde $\boldsymbol{\epsilon}_k$ é o erro de observação, com covariância $\mathbf{R}_k$.

%---------------------------------------------------------
\section{Etapas do Filtro de Kalman}
O filtro de Kalman opera em dois passos fundamentais: \emph{previsão} e \emph{atualização}.

\subsection*{1. Previsão (forecast)}
\[
\mathbf{x}_k^b = \mathbf{M}_{k-1}\mathbf{x}_{k-1}^a,
\]
\[
\mathbf{B}_k = \mathbf{M}_{k-1}\mathbf{A}_{k-1}\mathbf{M}_{k-1}^\top + \mathbf{Q}_{k-1}.
\]
Aqui, $\mathbf{x}_k^b$ é o estado previsto (background) e $\mathbf{B}_k$ a covariância de erro associada.

\subsection*{2. Atualização (análise)}
Quando as observações $\mathbf{y}_k$ tornam-se disponíveis, calcula-se a inovação:
\[
\mathbf{d}_k = \mathbf{y}_k - \mathbf{H}_k\mathbf{x}_k^b.
\]
Em seguida, aplica-se o ganho ótimo:
\begin{equation}
\mathbf{K}_k = \mathbf{B}_k \mathbf{H}_k^\top
(\mathbf{H}_k \mathbf{B}_k \mathbf{H}_k^\top + \mathbf{R}_k)^{-1},
\label{eq:gain-kalman}
\end{equation}
e atualiza-se a análise:
\[
\mathbf{x}_k^a = \mathbf{x}_k^b + \mathbf{K}_k \mathbf{d}_k.
\]
A covariância de erro é atualizada por:
\[
\mathbf{A}_k = (\mathbf{I} - \mathbf{K}_k \mathbf{H}_k)\mathbf{B}_k.
\]

Essas equações constituem o \textbf{Filtro de Kalman Linear}, que fornece a estimativa ótima em sentido de mínimos quadrados para sistemas lineares com erros gaussianos.

%---------------------------------------------------------
\section{Interpretação geométrica e probabilística}
O filtro de Kalman realiza, a cada ciclo, uma fusão de distribuições de probabilidade.  
O background é uma distribuição normal centrada em $\mathbf{x}_b$ com covariância $\mathbf{B}$; as observações, outra centrada em $\mathbf{y}$ com covariância $\mathbf{R}$.  
A análise é a distribuição resultante da combinação dessas duas informações, com variância reduzida (Figura~\ref{fig:kalman-geometry}).

\begin{figure}[h!]
\centering
\begin{tikzpicture}
  \begin{axis}[
    width=0.9\linewidth, height=6cm,
    xlabel={Estado $x$}, ylabel={Densidade $p(x)$},
    xmin=0, xmax=10, ymin=0, ymax=0.6,
    grid=both, grid style={densely dotted},
    legend style={at={(0.98,0.98)},anchor=north east,draw=none}
  ]
    \addplot+[domain=0:10, samples=300, thick] {exp(-((x-3)^2)/1.5)}; \addlegendentry{background}
    \addplot+[domain=0:10, samples=300, thick, dashed] {exp(-((x-7)^2)/0.8)}; \addlegendentry{observação}
    \addplot+[domain=0:10, samples=300, thick, color=brandA] {exp(-((x-5)^2)/0.5)}; \addlegendentry{análise}
  \end{axis}
\end{tikzpicture}
\caption{Interpretação probabilística do filtro de Kalman: a análise resulta da fusão entre a incerteza do modelo (background) e a das observações.}
\label{fig:kalman-geometry}
\end{figure}

%---------------------------------------------------------
\section{O Filtro de Kalman Estendido (EKF)}
Para sistemas não lineares, substituímos os operadores lineares por funções:
\[
\mathbf{x}_k = \mathcal{M}(\mathbf{x}_{k-1}) + \boldsymbol{\eta}_{k-1}, \qquad
\mathbf{y}_k = \mathcal{H}(\mathbf{x}_k) + \boldsymbol{\epsilon}_k.
\]
O \textbf{Filtro de Kalman Estendido (EKF)} lineariza essas funções em torno do estado atual:
\[
\mathbf{M}_k = \frac{\partial \mathcal{M}}{\partial \mathbf{x}} \bigg|_{\mathbf{x}_k^a}, \qquad
\mathbf{H}_k = \frac{\partial \mathcal{H}}{\partial \mathbf{x}} \bigg|_{\mathbf{x}_k^b}.
\]
O restante do algoritmo é idêntico ao filtro linear, mas a precisão depende da validade da linearização — o que pode ser problemático em sistemas altamente não lineares, como a atmosfera.

%---------------------------------------------------------
\section{O Filtro de Kalman por Conjunto (EnKF)}
O \textbf{Filtro de Kalman por Conjunto (Ensemble Kalman Filter)} foi desenvolvido para contornar as limitações computacionais do EKF.  
Em vez de armazenar a matriz $\mathbf{B}$ explicitamente, o EnKF representa as incertezas por um conjunto de amostras (\emph{ensemble}) do estado:
\[
\mathbf{B} \approx \frac{1}{N-1} \sum_{i=1}^{N} (\mathbf{x}_i^b - \bar{\mathbf{x}}^b)(\mathbf{x}_i^b - \bar{\mathbf{x}}^b)^\top.
\]
Cada membro do conjunto é atualizado com as mesmas equações de Kalman, usando as observações perturbadas:
\[
\mathbf{x}_{i,k}^a = \mathbf{x}_{i,k}^b + \mathbf{K}_k (\mathbf{y}_{i,k} - \mathbf{H}\mathbf{x}_{i,k}^b).
\]
O EnKF é amplamente utilizado em assimilação atmosférica e oceânica devido à sua escalabilidade e representação estatística natural das covariâncias.

%---------------------------------------------------------
\section{Aplicações meteorológicas}
O filtro de Kalman e suas variantes são aplicados em:
\begin{itemize}
  \item sistemas de previsão numérica do tempo (\textbf{NWP}), onde atualizam o estado atmosférico continuamente;
  \item assimilação de observações de satélite e superfície;
  \item previsão de estados oceânicos e hidrológicos;
  \item controle de qualidade de observações (detecção de outliers).
\end{itemize}
Em modelos globais, o EnKF é frequentemente acoplado a esquemas variacionais híbridos (3DVar/4DVar), resultando nos modernos sistemas \textbf{EnVar}.

%---------------------------------------------------------
\section{Síntese}
O filtro de Kalman representa a unificação dos conceitos de interpolação, mínimos quadrados e análise objetiva dentro de um arcabouço probabilístico dinâmico.  
Sua generalização — o EnKF — é hoje uma das ferramentas mais poderosas da assimilação numérica de dados geofísicos.  
No próximo capítulo, abordaremos a \emph{assimilação variacional}, mostrando como o mesmo problema pode ser formulado como a minimização de um funcional no espaço do modelo, levando aos métodos 3DVar e 4DVar.

% Fim do Capítulo 7
 % Cap. 7 — Filtro de Kalman, EKF, EnKF
%%=========================================================
% Capítulo 8 — A assimilação variacional (3DVar e 4DVar)
%=========================================================
\chapter{A assimilação variacional (3DVar e 4DVar)}
\label{ch:variacional}

\noindent\textbf{Resumo:}
Neste capítulo, apresentamos a formulação variacional da assimilação de dados, onde a análise é obtida como o mínimo de uma função custo.  
O método 3DVar resolve o problema em um instante de tempo, enquanto o 4DVar estende o ajuste para um intervalo temporal, utilizando a dinâmica do modelo.  
Essas abordagens constituem a base de muitos sistemas operacionais de previsão numérica do tempo.

%---------------------------------------------------------
\section{Motivação}
A assimilação variacional nasce do mesmo princípio do filtro de Kalman: combinar coerentemente background e observações, ponderados por seus erros.  
Mas, em vez de uma atualização recursiva, o método variacional formula o problema como uma \emph{minimização global} da função custo:
\begin{equation}
J(\mathbf{x}) = (\mathbf{x} - \mathbf{x}_b)^\top \mathbf{B}^{-1} (\mathbf{x} - \mathbf{x}_b)
+ (\mathbf{y} - \mathbf{H}\mathbf{x})^\top \mathbf{R}^{-1} (\mathbf{y} - \mathbf{H}\mathbf{x}).
\label{eq:J-3DVar}
\end{equation}
Essa forma é idêntica à função custo da análise estatística apresentada no Capítulo~\ref{ch:formulacao-estatistica}.  
A diferença é que, agora, resolvemos a minimização de forma iterativa e, no caso 4D, incorporamos a evolução temporal do modelo.

%---------------------------------------------------------
\section{O método 3DVar}
O \textbf{3DVar} (\emph{Three-Dimensional Variational Assimilation}) considera todas as observações em um instante fixo de tempo.  
O gradiente da função custo é:
\begin{equation}
\nabla J = \mathbf{B}^{-1} (\mathbf{x} - \mathbf{x}_b)
- \mathbf{H}^\top \mathbf{R}^{-1} (\mathbf{y} - \mathbf{H}\mathbf{x}).
\label{eq:grad3dvar}
\end{equation}
O mínimo é obtido quando $\nabla J = 0$, o que leva exatamente à solução de análise dada pela equação de Kalman estática:
\[
\mathbf{x}_a = \mathbf{x}_b + \mathbf{K} (\mathbf{y} - \mathbf{H}\mathbf{x}_b),
\]
com $\mathbf{K}$ definido como em \eqref{eq:kalman-gain}.  
A diferença é que, em 3DVar, não se calcula explicitamente $\mathbf{K}$; em vez disso, resolve-se \eqref{eq:J-3DVar} iterativamente, com métodos como \textbf{Conjugate Gradient} ou \textbf{L-BFGS}.

\begin{figure}[h!]
\centering
\begin{tikzpicture}
\begin{axis}[
  width=0.8\linewidth, height=6cm,
  xlabel={Estado $x$}, ylabel={Função custo $J(x)$},
  xmin=0, xmax=5, ymin=0, ymax=6,
  grid=both, grid style={densely dotted},
  legend style={at={(0.98,0.98)},anchor=north east,draw=none}
]
  \addplot+[domain=0:5, samples=200, thick] {(x-1)^2/2 + (2-x)^2};
  \addplot+[only marks,mark=*] coordinates {(1.4,0.68)}; 
  \node[anchor=west] at (axis cs:1.4,0.7) {$x_a$};
\end{axis}
\end{tikzpicture}
\caption{Interpretação geométrica da análise 3DVar: o ponto mínimo de $J(x)$ representa o estado mais provável, conciliando modelo e observações.}
\label{fig:3dvar-cost}
\end{figure}

%---------------------------------------------------------
\section{O método 4DVar}
O \textbf{4DVar} (\emph{Four-Dimensional Variational Assimilation}) estende o conceito para um período de tempo $[t_0, t_n]$.  
A função custo incorpora todas as observações disponíveis ao longo desse intervalo, aplicando a dinâmica do modelo como restrição:
\begin{equation}
J(\mathbf{x}_0) = (\mathbf{x}_0 - \mathbf{x}_b)^\top \mathbf{B}^{-1} (\mathbf{x}_0 - \mathbf{x}_b)
+ \sum_{k=0}^{n} (\mathbf{y}_k - \mathbf{H}_k \mathbf{M}_{k,0} \mathbf{x}_0)^\top
\mathbf{R}_k^{-1} (\mathbf{y}_k - \mathbf{H}_k \mathbf{M}_{k,0} \mathbf{x}_0),
\label{eq:J-4DVar}
\end{equation}
onde $\mathbf{M}_{k,0}$ é o modelo linearizado que propaga o estado inicial $\mathbf{x}_0$ até o tempo $t_k$.

Assim, o 4DVar procura o estado inicial que, ao ser evoluído pelo modelo, melhor ajusta todas as observações no período.

%---------------------------------------------------------
\section{O papel do operador adjunto}
A minimização de \eqref{eq:J-4DVar} requer o cálculo eficiente do gradiente $\nabla J$.  
Como o modelo é geralmente de alta dimensão, calcular derivadas diretas é inviável.  
Utiliza-se então o \textbf{operador adjunto} $\mathbf{M}^\top$, que propaga as sensibilidades de volta no tempo:
\[
\nabla J = \mathbf{B}^{-1} (\mathbf{x}_0 - \mathbf{x}_b)
- \sum_{k=0}^{n} \mathbf{M}_{k,0}^\top \mathbf{H}_k^\top \mathbf{R}_k^{-1}
(\mathbf{y}_k - \mathbf{H}_k \mathbf{M}_{k,0} \mathbf{x}_0).
\]
Esse cálculo adjunto permite a otimização iterativa mesmo em sistemas com milhões de variáveis, como os modelos atmosféricos globais.

\begin{figure}[h!]
\centering
\begin{tikzpicture}
\begin{axis}[
  width=0.9\linewidth, height=6cm,
  xlabel={tempo}, ylabel={diferença modelo–observação},
  xmin=0, xmax=10, ymin=-3, ymax=3,
  grid=both, grid style={densely dotted}
]
  \addplot+[domain=0:10, samples=200, thick] {2*sin(deg(0.7*x)) - 0.5*x};
  \addplot+[domain=0:10, samples=200, thick, dashed, color=brandA] {2*sin(deg(0.7*x))};
  \legend{modelo, observações}
\end{axis}
\end{tikzpicture}
\caption{O 4DVar ajusta o estado inicial para que a trajetória do modelo se alinhe com as observações ao longo de todo o intervalo.}
\label{fig:4dvar}
\end{figure}

%---------------------------------------------------------
\section{Comparação entre 3DVar, 4DVar e EnKF}
\begin{center}
\begin{tabular}{lccc}
\toprule
\textbf{Característica} & \textbf{3DVar} & \textbf{4DVar} & \textbf{EnKF} \\
\midrule
Dependência temporal & Instantânea & Intervalo $[t_0,t_n]$ & Sequencial \\
Linearização do modelo & Não & Necessária & Não \\
Representação de erro & Estatística fixa ($B$) & Linearizada via modelo & Amostrada (ensemble) \\
Custo computacional & Moderado & Elevado & Escalável \\
Uso do adjunto & Não & Sim & Não \\
\bottomrule
\end{tabular}
\end{center}

Cada método tem suas vantagens:
\begin{itemize}
  \item o \textbf{3DVar} é simples e eficiente, adequado para ciclos rápidos;
  \item o \textbf{4DVar} fornece a melhor coerência temporal, ao custo de maior complexidade;
  \item o \textbf{EnKF} é flexível e ideal para modelos complexos e não lineares.
\end{itemize}

%---------------------------------------------------------
\section{Síntese}
O método variacional reformula a assimilação como um problema de otimização no espaço do modelo.  
Enquanto o Filtro de Kalman atualiza explicitamente o estado, o 4DVar busca o estado inicial ótimo cujo desenvolvimento temporal é mais consistente com as observações.  
A equivalência entre o filtro e o método variacional foi demonstrada por Lorenc (1986), unificando os dois paradigmas sob a mesma base estatística.

No próximo capítulo, exploraremos os \emph{métodos híbridos}, que combinam as vantagens do 4DVar e do EnKF para obter análises consistentes, dinâmicas e estatisticamente robustas.

% Fim do Capítulo 8
 % Cap. 8 — 3DVar e 4DVar
%%=========================================================
% Capítulo 9 — Métodos híbridos e o futuro da assimilação de dados
%=========================================================
\chapter{Métodos híbridos e o futuro da assimilação de dados}
\label{ch:hibridos}

\noindent\textbf{Resumo:}
Este capítulo apresenta os métodos híbridos de assimilação de dados, que combinam as vantagens do 4DVar (consistência dinâmica) e do EnKF (representação estatística).  
Essas abordagens modernas constituem a base dos sistemas de assimilação atuais em centros meteorológicos globais.  
Também discutimos as perspectivas futuras: o uso de inteligência artificial, aprendizado de máquina e assimilação acoplada no Sistema Terrestre.

%---------------------------------------------------------
\section{A motivação para métodos híbridos}
O 4DVar fornece análises coerentes no tempo, mas depende de operadores adjuntos e de uma matriz de covariância fixa ($\mathbf{B}$).  
Já o EnKF atualiza estatísticas dinamicamente, mas não garante uma coerência temporal contínua.  
Os \textbf{métodos híbridos} surgem para reunir o melhor dos dois mundos:
\begin{itemize}
  \item do 4DVar, herdam a estrutura variacional e o uso do modelo dinâmico;
  \item do EnKF, herdam a estimativa amostral das covariâncias de erro.
\end{itemize}

Dessa combinação nascem os esquemas conhecidos como \textbf{3DEnVar} e \textbf{4DEnVar}.

%---------------------------------------------------------
\section{O método 3DEnVar}
O \textbf{3DEnVar} substitui a matriz de covariância fixa do 3DVar por uma combinação entre uma matriz climatológica ($\mathbf{B}_{\text{clim}}$) e uma matriz amostral derivada de um ensemble ($\mathbf{B}_{\text{ens}}$):
\begin{equation}
\mathbf{B}_{\text{híbrida}} = \alpha \mathbf{B}_{\text{clim}} + (1 - \alpha)\mathbf{B}_{\text{ens}},
\label{eq:bhibrida}
\end{equation}
onde $0 \leq \alpha \leq 1$ controla o peso relativo entre os componentes.  
Essa mistura permite representar covariâncias dependentes do fluxo (estado atmosférico atual) sem perder estabilidade estatística.

A função custo resultante é:
\[
J(\mathbf{x}) = (\mathbf{x} - \mathbf{x}_b)^\top \mathbf{B}_{\text{híbrida}}^{-1} (\mathbf{x} - \mathbf{x}_b)
+ (\mathbf{y} - \mathbf{H}\mathbf{x})^\top \mathbf{R}^{-1} (\mathbf{y} - \mathbf{H}\mathbf{x}).
\]
O método é computacionalmente eficiente, pois não requer operadores adjuntos e pode ser executado em paralelo para cada membro do ensemble.

%---------------------------------------------------------
\section{O método 4DEnVar}
O \textbf{4DEnVar} estende o conceito do 3DEnVar, incorporando a evolução temporal do modelo, mas sem usar o adjunto.  
Em vez disso, utiliza um conjunto de trajetórias de previsão (ensemble forecasts) para representar a evolução das covariâncias no tempo:
\[
\mathbf{B}_k^{\text{ens}} \approx \frac{1}{N-1} \sum_{i=1}^N (\mathbf{x}_{i,k}^b - \bar{\mathbf{x}}_k^b)
(\mathbf{x}_{i,k}^b - \bar{\mathbf{x}}_k^b)^\top.
\]
Dessa forma, o 4DEnVar captura a variabilidade espaço-temporal da atmosfera sem a necessidade de derivadas analíticas do modelo.

O resultado é um sistema de assimilação consistente, paralelo e altamente escalável — ideal para aplicações em larga escala.

%---------------------------------------------------------
\section{Comparação entre as abordagens}
\begin{center}
\begin{tabular}{lcccc}
\toprule
\textbf{Método} & \textbf{Covariância} & \textbf{Adjunto} & \textbf{Evolução temporal} & \textbf{Custo} \\
\midrule
3DVar & Fixa (climatológica) & Não & Instante & Baixo \\
4DVar & Fixa (via modelo) & Sim & Contínua & Alto \\
EnKF & Amostral (ensemble) & Não & Sequencial & Moderado \\
3DEnVar & Híbrida (fixa+ensemble) & Não & Instante & Moderado \\
4DEnVar & Híbrida (ensemble temporal) & Não & Intervalo & Alto \\
\bottomrule
\end{tabular}
\end{center}

Os métodos híbridos permitem incorporar a dinâmica atmosférica (como no 4DVar) e as estatísticas dependentes do fluxo (como no EnKF), com desempenho adequado para execução operacional em supercomputadores modernos.

%---------------------------------------------------------
\section{Aplicações e avanços recentes}
Os esquemas híbridos já são amplamente usados em centros operacionais como ECMWF, NCEP, UK Met Office e JMA.  
Eles oferecem melhor coerência entre variáveis, redução de ruído espúrio e maior capacidade de assimilação de dados de satélite e radar.

Além disso, novas abordagens buscam reduzir o custo e aumentar a robustez:
\begin{itemize}
  \item \textbf{EnVar local}: aplica a assimilação em subdomínios independentes, reduzindo o custo computacional;
  \item \textbf{4DEnVar incremental}: realiza a minimização em incrementos lineares de estado;
  \item \textbf{EnVar acoplado}: integra dados atmosféricos, oceânicos, terrestres e de gelo em um único sistema.
\end{itemize}

\begin{figure}[h!]
\centering
\begin{tikzpicture}
\begin{axis}[
  width=0.9\linewidth, height=6cm,
  xlabel={Tempo}, ylabel={Redução do erro RMS},
  xmin=0, xmax=10, ymin=0, ymax=1,
  grid=both, grid style={densely dotted},
  legend style={at={(0.98,0.98)},anchor=north east,draw=none}
]
  \addplot+[domain=0:10, samples=200, thick] {exp(-0.3*x)};
  \addplot+[domain=0:10, samples=200, thick, dashed] {exp(-0.5*x)};
  \addplot+[domain=0:10, samples=200, thick, color=brandA] {exp(-0.7*x)};
  \legend{3DVar, EnKF, 4DEnVar}
\end{axis}
\end{tikzpicture}
\caption{Comparação esquemática da redução do erro RMS ao longo do tempo. O 4DEnVar atinge melhor desempenho devido à combinação entre coerência temporal e covariâncias dinâmicas.}
\label{fig:rms-comparison}
\end{figure}

%---------------------------------------------------------
\section{O futuro da assimilação de dados}
A próxima geração de sistemas de assimilação está avançando em três direções principais:

\subsection*{1. Assimilação acoplada}
Os sistemas acoplados Atmosfera–Oceano–Terra–Gelo (\emph{Earth System Data Assimilation}) tratam as interações entre componentes do Sistema Terrestre, buscando consistência física global.

\subsection*{2. Métodos assistidos por IA}
Modelos de aprendizado profundo vêm sendo explorados para:
\begin{itemize}
  \item emular operadores de observação ($\mathbf{H}$);
  \item corrigir erros sistemáticos do modelo ($\mathbf{M}$);
  \item estimar diretamente o ganho de Kalman ou o gradiente de $J$;
  \item criar redes híbridas (físico–estatísticas) para acelerar a assimilação.
\end{itemize}
Esses sistemas combinam a capacidade de aprendizado dos modelos neurais com o rigor físico das equações dinâmicas.

\subsection*{3. Assimilação em tempo real e HPC exascale}
A transição para arquiteturas massivamente paralelas e aprendizado distribuído permitirá que a assimilação opere em tempo quase real, integrando bilhões de observações diárias — inclusive de sensores IoT, drones e satélites geoestacionários de alta resolução.

%---------------------------------------------------------
\section{Síntese}
Os métodos híbridos representam a convergência das abordagens estatística (EnKF) e variacional (4DVar).  
Eles simbolizam a maturidade da assimilação de dados: uma ciência que combina teoria de estimação, física atmosférica e ciência da computação.  
O futuro da área aponta para sistemas cada vez mais integrados, inteligentes e acoplados, capazes de transformar observações brutas em conhecimento preditivo sobre o planeta.

% Fim do Capítulo 9
 % Cap. 9 — Métodos híbridos (3D/4DEnVar)
%%=========================================================
% Capítulo 10 — Sistemas operacionais e exemplos reais
%=========================================================
\chapter{Sistemas operacionais e exemplos reais}
\label{ch:sistemas-operacionais}

\noindent\textbf{Resumo:}
Este capítulo descreve como os princípios teóricos da assimilação são implementados em sistemas operacionais reais.
São apresentados os principais esquemas usados em meteorologia — GSI, JEDI e LETKF — e o ciclo de assimilação típico que integra observações de satélite, superfície e sondagens.
Discute-se também o papel das observações convencionais e remotas, e como elas alimentam os modelos numéricos de previsão.

%---------------------------------------------------------
\section{O ciclo de assimilação}
Um \emph{ciclo de assimilação} representa a sequência de etapas que transforma dados observacionais em um estado inicial coerente para o modelo.
Cada ciclo compreende:
\begin{enumerate}
  \item previsão de curto prazo (\emph{forecast}) até o próximo tempo de análise;
  \item coleta e pré-processamento de observações;
  \item assimilação das observações no modelo;
  \item geração do campo analisado;
  \item reinício da próxima previsão.
\end{enumerate}
Em operação, esse ciclo é executado a cada 6 horas ou menos, garantindo que o sistema de previsão permaneça atualizado com as condições atmosféricas reais.

\begin{figure}[h!]
\centering
\begin{tikzpicture}[>=latex, node distance=1.8cm, every node/.style={align=center}]
  \node[draw,rounded corners,fill=gray!10,minimum width=3cm] (fc) {Previsão};
  \node[draw,rounded corners,fill=gray!10,below of=fc] (obs) {Coleta de\\observações};
  \node[draw,rounded corners,fill=gray!10,below of=obs] (assim) {Assimilação};
  \node[draw,rounded corners,fill=gray!10,below of=assim] (anal) {Análise};
  \node[draw,rounded corners,fill=gray!10,below of=anal] (next) {Próxima previsão};
  \draw[->,thick] (fc) -- (obs);
  \draw[->,thick] (obs) -- (assim);
  \draw[->,thick] (assim) -- (anal);
  \draw[->,thick] (anal) -- (next);
  \draw[->,thick,bend left=50] (next.east) to node[right]{reinício} (fc.east);
\end{tikzpicture}
\caption{Esquema conceitual do ciclo operacional de assimilação de dados.}
\label{fig:cycle}
\end{figure}

%---------------------------------------------------------
\section{O sistema GSI}
O \textbf{Gridpoint Statistical Interpolation} (GSI) é um dos sistemas mais difundidos de assimilação variacional.
Desenvolvido pelo NCEP (EUA), ele implementa as formas 3DVar e 4DEnVar.
Sua função custo segue a equação~\eqref{eq:J-3DVar}, e o processo é resolvido iterativamente usando o método de gradiente conjugado.

O GSI é amplamente utilizado no Brasil (INPE/CPTEC, SMNA) e em instituições internacionais (NOAA, NASA, JCSDA).
Ele assimila observações de superfície, sondagens, dados de satélite (radiâncias), vento derivado de imagens, GPSRO e perfis de umidade.
Cada tipo de observação é representado por um \emph{operador de observação} específico — parte essencial do código.

%---------------------------------------------------------
\section{O sistema JEDI}
O \textbf{Joint Effort for Data Assimilation Integration} (JEDI) é uma plataforma modular e moderna desenvolvida pelo \emph{Joint Center for Satellite Data Assimilation} (JCSDA).
Seu objetivo é unificar a assimilação para diferentes modelos (atmosfera, oceano, superfície) em uma estrutura comum orientada a objetos em C++/Fortran, com configuração em YAML.
Entre suas principais características:
\begin{itemize}
  \item arquitetura genérica e escalável;
  \item suporte a assimilação variacional e \emph{ensemble};
  \item integração com sistemas de observação via OOPS (Object-Oriented Prediction System);
  \item compilação automatizada via \texttt{spack-stack}.
\end{itemize}
O JEDI implementa os métodos 3DVar, 4DVar, EnKF e híbridos EnVar, tornando-se a principal base para o desenvolvimento futuro de sistemas comunitários como o \textbf{MONAN} no Brasil.

%---------------------------------------------------------
\section{O sistema LETKF}
O \textbf{Local Ensemble Transform Kalman Filter} (LETKF) é um esquema baseado em EnKF, mas aplicado localmente em cada ponto de grade.
Isso permite paralelismo massivo e menor custo de comunicação.
Cada subdomínio realiza sua própria análise com base em um pequeno subconjunto de observações próximas:
\[
\mathbf{x}_a^l = \mathbf{x}_b^l + \mathbf{K}_l (\mathbf{y}^l - \mathbf{H}_l \mathbf{x}_b^l),
\]
onde o índice $l$ representa uma região local.
A análise global é então recomposta pela junção contínua dos resultados locais.

O LETKF é utilizado em centros como JMA (Japão), ECMWF e em pesquisas no Brasil (CPTEC/INPE, IAG/USP), demonstrando excelente desempenho em modelos de alta resolução.

%---------------------------------------------------------
\section{Tipos de observações assimiladas}
Os sistemas operacionais incorporam uma ampla gama de dados:
\begin{itemize}
  \item \textbf{Observações de superfície:} estações meteorológicas, boias oceânicas, aeroportos (SYNOP, METAR, SHIP, BUOY);
  \item \textbf{Sondagens:} radiossondas, dropsondes, balões meteorológicos;
  \item \textbf{Satélites:} radiâncias (ATMS, AMSU-A, IASI, GOES-16), perfis de vento e umidade, GPSRO;
  \item \textbf{Radar meteorológico:} velocidade radial e refletividade;
  \item \textbf{Aviação e reanálises:} dados de aeronaves (AMDAR) e produtos compostos.
\end{itemize}
Essas observações passam por controles de qualidade e normalização antes de serem assimiladas.
O equilíbrio entre cobertura espacial e precisão é fundamental para o desempenho global do sistema.

%---------------------------------------------------------
\section{Exemplo prático: ciclo de 6 horas}
Em um sistema global típico (como o BAM + GSI do CPTEC), o ciclo de assimilação segue:
\begin{enumerate}
  \item \textbf{T\textsubscript{0}} – Coleta de observações entre T–3 h e T+3 h;
  \item \textbf{Análise} – Execução do GSI com os dados do período;
  \item \textbf{Integração} – Modelo BAM executado por 6 h;
  \item \textbf{Saída} – Resultado usado como background do próximo ciclo.
\end{enumerate}
Esse processo é automatizado em ambientes HPC com gerenciadores de filas (SLURM, PBS) e scripts de controle.

%---------------------------------------------------------
\section{Síntese}
Os sistemas operacionais (GSI, JEDI, LETKF) materializam a teoria da assimilação em fluxos de produção complexos e massivamente paralelos.
O sucesso da previsão numérica moderna depende da integração harmônica entre o modelo, o pré-processamento das observações e o ciclo de assimilação.
Esses sistemas são hoje o núcleo operacional de previsão e pesquisa em meteorologia.

% Fim do Capítulo 10
 % Cap. 10 — Sistemas operacionais: GSI, JEDI, LETKF
%%=========================================================
% Capítulo 11 — O futuro da assimilação de dados
%=========================================================
\chapter{O futuro da assimilação de dados}
\label{ch:futuro}

\noindent\textbf{Resumo:}
Este capítulo explora as fronteiras emergentes da assimilação de dados (AD), com ênfase na integração entre métodos híbridos, aprendizado de máquina e sistemas acoplados do Sistema Terrestre.
Também são discutidos os desafios específicos da assimilação em regiões tropicais e o papel dos novos sensores na melhoria das condições iniciais e da representação física dos processos atmosféricos.

%---------------------------------------------------------
\section{A evolução contínua da assimilação de dados}
A assimilação de dados evoluiu de simples métodos de interpolação para sofisticados sistemas probabilísticos capazes de fundir bilhões de observações diárias.
Entretanto, mesmo após décadas de avanços, persistem desafios relacionados à complexidade dos modelos, aos erros correlacionados e à representação dos processos não lineares.
O futuro da AD depende de estratégias híbridas, uso de inteligência artificial e uma visão integrada do Sistema Terrestre.

%---------------------------------------------------------
\section{Métodos híbridos e aprendizado de máquina}
Os métodos híbridos representam a síntese entre abordagens variacionais (4DVar) e estatísticas (EnKF).
A próxima geração de sistemas está incorporando componentes baseados em \emph{machine learning} (ML) para resolver limitações persistentes:
\begin{itemize}
  \item aprendizado dinâmico de matrizes de erro de background ($\mathbf{B}$) a partir de ensembles históricos;
  \item redes neurais capazes de estimar operadores de observação ($\mathbf{H}$) e jacobianos aproximados;
  \item correção de erros sistemáticos de modelo por redes de aprendizado residual;
  \item assimilação \emph{end-to-end}, onde o modelo e a assimilação são treinados de forma conjunta.
\end{itemize}

O uso de ML reduz o custo computacional de operadores complexos, especialmente radiativos, e permite explorar informações observacionais ainda subutilizadas.

\begin{figure}[h!]
\centering
\begin{tikzpicture}[>=latex]
  \node[draw,rounded corners,fill=blue!10,minimum width=3cm] (ens) {Ensemble Kalman};
  \node[draw,rounded corners,fill=green!10,right=2cm of ens] (ml) {Rede Neural};
  \node[draw,rounded corners,fill=orange!10,below=1.5cm of $(ens)!0.5!(ml)$] (hibrido) {Método Híbrido};
  \draw[->,thick] (ens) -- (hibrido);
  \draw[->,thick] (ml) -- (hibrido);
\end{tikzpicture}
\caption{Convergência entre assimilação estatística e aprendizado de máquina.}
\label{fig:hibrido-ml}
\end{figure}

Essa convergência não substitui a física do modelo — ela a complementa, permitindo que as técnicas de ML aprendam padrões residuais e não lineares, mantendo o rigor físico.

%---------------------------------------------------------
\section{Assimilação acoplada do Sistema Terrestre}
O avanço da modelagem acoplada (Atmosfera–Oceano–Superfície–Criosfera) trouxe a necessidade de assimilação acoplada — ou \textbf{Earth System Data Assimilation (ESDA)}.
Nessa abordagem, a AD considera as interações entre componentes do sistema:
\begin{itemize}
  \item fluxos de calor, umidade e momentum entre oceano e atmosfera;
  \item acoplamento entre umidade do solo, vegetação e trocas radiativas;
  \item retroalimentação entre gelo marinho, temperatura e circulação global.
\end{itemize}

A assimilação acoplada melhora a coerência entre variáveis e reduz discrepâncias entre subsistemas.
No contexto brasileiro, sistemas como o MONAN incorporam essa filosofia, integrando modelos atmosféricos (MPAS), oceânicos (HYCOM) e de superfície continental (SSiB).

%---------------------------------------------------------
\section{Desafios da assimilação em regiões tropicais}
As regiões tropicais representam um dos maiores desafios da assimilação moderna.
Os sistemas convencionais foram otimizados para latitudes médias, onde as ondas baroclínicas dominam e os erros são mais gaussianos.
Nos trópicos:
\begin{itemize}
  \item predominam movimentos convectivos rápidos e não lineares;
  \item as observações são escassas em altitude e densas em superfície;
  \item a relação entre variáveis (temperatura, umidade, vento) é fortemente acoplada.
\end{itemize}
Essas características exigem técnicas adaptativas, como:
\begin{enumerate}
  \item aumento da frequência de assimilação (por exemplo, a cada hora);
  \item filtros não lineares (EnKF estocástico, PF);
  \item aprendizado de correlações não gaussianas via ML;
  \item uso de dados geoestacionários de alta resolução.
\end{enumerate}

\begin{figure}[h!]
\centering
\begin{tikzpicture}
\begin{axis}[
  width=0.9\linewidth, height=5.5cm,
  xlabel={Latitude}, ylabel={Cobertura observacional},
  xmin=-90, xmax=90, ymin=0, ymax=1.0,
  grid=both, grid style={densely dotted},
  xtick={-90,-60,-30,0,30,60,90},
  legend style={at={(0.97,0.97)},anchor=north east,draw=none}
]
  \addplot+[smooth,thick,domain=-90:90,samples=200] {0.7+0.2*sin(deg(x/45))};
  \addplot+[smooth,thick,dashed,domain=-90:90,samples=200] {0.4+0.4*cos(deg(x/90))};
  \legend{Satélite Polar, Satélite Geoestacionário}
\end{axis}
\end{tikzpicture}
\caption{Cobertura observacional média por latitude. Nos trópicos, os satélites geoestacionários (como GOES-16) são essenciais para compensar a falta de sondagens.}
\label{fig:cobertura-tropicos}
\end{figure}

%---------------------------------------------------------
\section{O papel dos novos sensores e observações emergentes}
O avanço instrumental abre uma nova era para a assimilação:
\begin{itemize}
  \item \textbf{Satélites de nova geração:} GOES-R, MTG, METOP-SG e JPSS fornecem medições hiperespectrais e temporais de alta resolução;
  \item \textbf{Microssatélites e constelações:} missões CubeSat (TROPICS, CYGNSS) aumentam a densidade observacional nos trópicos;
  \item \textbf{Sensores de superfície inteligentes:} redes IoT meteorológicas e agrícolas permitem assimilação de dados locais em tempo real;
  \item \textbf{Radar de dupla polarização:} fornece observações tridimensionais de hidrometeoros e vento vertical;
  \item \textbf{GNSS-RO e interferometria:} melhoram o perfilamento da atmosfera em escala global.
\end{itemize}

Essas novas fontes de dados impõem desafios de calibração e controle de qualidade, mas ampliam consideravelmente o potencial preditivo dos modelos numéricos.

%---------------------------------------------------------
\section{Síntese e perspectivas}
O futuro da assimilação de dados caminha para sistemas:
\begin{itemize}
  \item \textbf{acoplados e multiescalares}, integrando componentes do Sistema Terrestre;
  \item \textbf{inteligentes}, com aprendizado adaptativo de erros e relações entre variáveis;
  \item \textbf{massivamente paralelos}, explorando arquiteturas de HPC e computação exascale;
  \item \textbf{autônomos}, capazes de operar continuamente com entrada de novos sensores.
\end{itemize}

A fronteira entre modelo e observação tende a desaparecer: cada modelo será, em essência, um assimilador dinâmico, ajustando-se permanentemente ao mundo real.
Essa visão coloca a assimilação de dados como um pilar central da ciência de previsão — o elo entre observação, teoria e simulação numérica.

% Fim do Capítulo 11
 % Cap. 11 — Futuro da AD: IA, acoplado, trópicos, sensores

%======================================================================
% Parte final — Materiais de apoio
%======================================================================
\backmatter

% Listas de siglas e glossário
\printglossary[type=\acronymtype,title={Lista de Siglas}]
\printglossary[title={Glossário}]

% Referências bibliográficas
\printbibliography[title={Referências}]

% Índice remissivo
\printindex

% Apêndices
\appendix
\chapter{Apêndice A: Notação e símbolos}
\chapter{Apêndice B: Exercícios comentados}

%======================================================================
\end{document}
%======================================================================


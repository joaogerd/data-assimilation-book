\documentclass{article}

% ------ fontes e layout básicos ------
\usepackage[T1]{fontenc}
\usepackage[utf8]{inputenc}
\usepackage{lmodern}
\usepackage[margin=0mm]{geometry}

% ------ TikZ ------
\usepackage{tikz}
\usetikzlibrary{
  arrows.meta,
  calc,
  decorations.text,
  decorations.pathmorphing,
  positioning
}

% ------ paleta (tons "met oce/teal") ------
\definecolor{c1}{RGB}{ 28, 53, 77}   % faixa/título
\definecolor{c2}{RGB}{ 62, 97,127}   % contornos fortes
\definecolor{c3}{RGB}{104,182,182}   % contornos médios
\definecolor{c4}{RGB}{150,210,210}   % contornos leves
\definecolor{c5}{RGB}{ 95,162,162}   % detalhes
\definecolor{used}{RGB}{ 40,160, 60} % obs usadas
\definecolor{rej}{RGB}{200, 70, 70}  % obs rejeitadas
\definecolor{bg}{RGB}{245,248,250}   % fundo

\pagestyle{empty}

\begin{document}
\thispagestyle{empty}

\begin{tikzpicture}[remember picture, overlay]
  % ===== fundo geral =====
  \fill[bg] (current page.south west) rectangle (current page.north east);

  % ===== moldura discreta =====
  \draw[line width=5pt, draw=c4!30]
    ([xshift=3mm,yshift=3mm]current page.south west)
    rectangle
    ([xshift=-3mm,yshift=-3mm]current page.north east);

  % ===== faixa lateral para título =====
  \path let \p1 = (current page.north west), \p2 = (current page.south west) in
    coordinate (LW) at (\x1,\y1)
    coordinate (SW) at (\x2,\y2);
  % largura da faixa:
  \def\SideW{0.33*\paperwidth}
  \fill[c1] (LW) rectangle ++(\SideW,-\paperheight);

  % ===== bloco do campo com isolinhas (clipe à área direita) =====
  \begin{scope}
    \clip ([xshift=\SideW]current page.north west) rectangle (current page.south east);

    % grade leve (lat-lon estilizada)
    \foreach \x in {0.08,0.16,...,0.96} {
      \draw[c4!25, line width=0.4pt] ($(current page.south west)!{\x}!(current page.south east)$) -- ++(0,\paperheight);
    }
    \foreach \y in {0.08,0.16,...,0.96} {
      \draw[c4!25, line width=0.4pt] ($(current page.south west)!{\y}!(current page.north west)$) -- ++(\paperwidth,0);
    }

    % "núcleo" (tipo um vórtice/baixa) para dar referência visual
    \coordinate (C) at ($(current page.center)+(0.18\paperwidth, -0.12\paperheight)$);

    % isolinhas (3 níveis, mais rugosas conforme se afastam)
    % nível interno (liso)
    \draw[c2, line width=2pt]
      plot [smooth cycle, tension=0.9]
      coordinates {
        ($(C)+(-4.0,  1.2)$)
        ($(C)+(-1.2,  2.2)$)
        ($(C)+( 1.0,  2.0)$)
        ($(C)+( 3.2,  0.8)$)
        ($(C)+( 2.6, -1.4)$)
        ($(C)+( 0.6, -2.4)$)
        ($(C)+(-2.6,-1.8)$)
      };

    % nível médio (um pouco “rough”)
    \draw[c3, line width=1.6pt, decorate,
          decoration={random steps, segment length=10pt, amplitude=2pt}]
      plot [smooth cycle, tension=0.85]
      coordinates {
        ($(C)+(-6.4,  1.6)$)
        ($(C)+(-2.2,  3.0)$)
        ($(C)+( 1.4,  3.0)$)
        ($(C)+( 4.4,  1.6)$)
        ($(C)+( 4.0, -2.2)$)
        ($(C)+( 0.8, -3.4)$)
        ($(C)+(-3.6,-2.6)$)
      };

    % nível externo (mais rugoso)
    \draw[c4, line width=1.2pt, decorate,
          decoration={random steps, segment length=12pt, amplitude=3pt}]
      plot [smooth cycle, tension=0.82]
      coordinates {
        ($(C)+(-8.0,  2.2)$)
        ($(C)+(-3.0,  4.0)$)
        ($(C)+( 2.0,  4.0)$)
        ($(C)+( 6.2,  2.0)$)
        ($(C)+( 5.6, -3.0)$)
        ($(C)+( 0.6, -4.4)$)
        ($(C)+(-4.8, -3.4)$)
      };

    % observações (usadas e rejeitadas)

    % Lista de deslocamentos x/y (sem $...$ aqui)
    \foreach \dx/\dy in {-2.8/1.0, -1.4/1.8, 0.6/1.6, 2.2/0.4, 1.8/-0.8, -0.6/-1.2, -2.0/-0.6}{
  
      % Cria uma coordenada P = C + (dx,dy) usando calc corretamente:
      \path coordinate (P) at ($(C)+(\dx,\dy)$);
  
      % Ponto
      \fill[used] (P) circle (2.2pt);
  
      % Cruz
      \draw[used, line width=0.8pt] ($(P)+(-0.14,0)$) -- ($(P)+(0.14,0)$);
      \draw[used, line width=0.8pt] ($(P)+(0,-0.14)$) -- ($(P)+(0,0.14)$);
    }
  
    \foreach \dx/\dy in{ 
        -3.6/2.6,
         3.2/1.8,
         3.6/-1.8,
        -1.6/-2.4
    }{
        \path coordinate (P) at ($(C)+(\dx,\dy)$);
        \draw[rej, line width=1pt] ($(P)+(-0.18,-0.18)$) -- ($(P)+(0.18,0.18)$);
        \draw[rej, line width=1pt] ($(P)+(-0.18,0.18)$) -- ($(P)+(0.18,-0.18)$);
    }



    % seta “background -> análise”
    \draw[-{Latex[length=8pt,width=6pt]}, line width=2pt, c5]
      ($(C)+(-0.09\paperwidth, 0.11\paperheight)$)
      to[bend left=15]
      node[midway, above right=2pt, c5!60!black, font=\sffamily\small]{background $\to$ an\'alise}
      ($(C)+(0.06\paperwidth, -0.02\paperheight)$);

%    % equações ao longo de caminhos curvos
%    % caminho 1 – função custo
%    \path[c3, line width=1pt, name path=Jpath]
%      ($(C)+(-0.22\paperwidth, 0.18\paperheight)$)
%      to[out=-20, in=160]
%      ($(C)+(0.10\paperwidth, 0.08\paperheight)$)
%      to[out=-20, in=160]
%      ($(C)+(0.28\paperwidth,-0.04\paperheight)$);
%
%    \draw[c3, line width=1pt, postaction={
%      decorate, decoration={
%        text along path,
%        text align=center,
%        raise=1.8ex,
%        text={|\sffamily\footnotesize\color{c3!90!black}|
%          $\displaystyle
%          J(\mathbf{x})=\tfrac{1}{2}(\mathbf{x}-\mathbf{x}^{b})^{\!T}\mathbf{B}^{-1}(\mathbf{x}-\mathbf{x}^{b})
%          +\tfrac{1}{2}(\mathbf{y}-\mathbf{H}\mathbf{x})^{\!T}\mathbf{R}^{-1}(\mathbf{y}-\mathbf{H}\mathbf{x})
%          $}}}]
%      (Jpath);
%
%    % caminho 2 – equação de análise
%    \path[c2, line width=1.2pt, name path=Apath]
%      ($(C)+(-0.18\paperwidth,-0.18\paperheight)$)
%      to[out=30, in=200]
%      ($(C)+(-0.02\paperwidth,-0.04\paperheight)$)
%      to[out=10, in=170]
%      ($(C)+(0.18\paperwidth, 0.02\paperheight)$);
%
%    \draw[c2, line width=1.2pt, postaction={
%      decorate, decoration={
%        text along path,
%        text align=center,
%        raise=1.8ex,
%        text={|\sffamily\footnotesize\color{c2}!90!black|
%          $\displaystyle
%          \mathbf{x}^{a}=\mathbf{x}^{b}+\mathbf{K}\bigl(\mathbf{y}-\mathbf{H}\mathbf{x}^{b}\bigr),\quad
%          \mathbf{K}=\mathbf{B}\mathbf{H}^{T}\bigl(\mathbf{H}\mathbf{B}\mathbf{H}^{T}+\mathbf{R}\bigr)^{-1}
%          $}}}]
%      (Apath);

  \end{scope}
%
%  % ===== Título/Subtítulo/Autor (na faixa) =====
%  \node[anchor=west, text=white, font=\sffamily\bfseries, align=left]
%    at ([xshift=10mm,yshift=-22mm]current page.north west)
%    {\fontsize{30}{36}\selectfont Introdu\c c\~ao \`a Assimila\c c\~ao de Dados\\[-1mm]
%     \rule{0.86\SideW}{1pt}\\[1mm]
%     \fontsize{14}{18}\selectfont Fundamentos, pr\'atica e perspectivas};
%
%  \node[anchor=west, text=white, font=\sffamily, align=left]
%    at ([xshift=10mm,yshift=-0.88\paperheight]current page.north west)
%    {\large Jo\~ao Gerd Zell de Mattos};
%
%  % ===== legenda pequena (opcional) =====
%  \node[anchor=south east, font=\sffamily\scriptsize, align=right, c2!80!black]
%    at ([xshift=-6mm,yshift=6mm]current page.south east)
%    {Isolinhas estilizadas, pontos de observa\c c\~ao\\
%     (verde: usadas; vermelho: rejeitadas)};

\end{tikzpicture}

\end{document}


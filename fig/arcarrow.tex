% ==============================================================
%  Macro e estilo do arco em seta (\arcarrow)
%  >> Inclua este arquivo com: % ==============================================================
%  Macro e estilo do arco em seta (\arcarrow)
%  >> Inclua este arquivo com: % ==============================================================
%  Macro e estilo do arco em seta (\arcarrow)
%  >> Inclua este arquivo com: % ==============================================================
%  Macro e estilo do arco em seta (\arcarrow)
%  >> Inclua este arquivo com: \input{fig/arcarrow.tex}
% ==============================================================
% --- TikZ: bibliotecas necessárias ---
\usetikzlibrary{decorations.text,positioning, shadows}
\usetikzlibrary{arrows.meta,calc} % (calc para coordenadas polares/expressões)
% --------------------------------------------------------------
% Estilo do texto desenhado AO LONGO DO ARCO (interno às setas)
% --------------------------------------------------------------
\newcommand*{\mytextstyle}{\sffamily\footnotesize\bfseries\color{black!85}}

% --------------------------------------------------------------
% MACRO PRINCIPAL: \arcarrow
% Desenha UMA seta em arco com texto curvado.
%
% Assinatura:
%   \arcarrow{rin}{rmid}{rout}{angIni}{angFim}{tip}{tikz-options}{texto}
%
% - rin, rmid, rout: raios interno, médio (texto) e externo da seta.
% - angIni, angFim:  ângulos em graus (orientação padrão do TikZ).
% - tip:              avanço (em graus) no raio médio para formar a “ponta”.
% - tikz-options:     opções passadas ao \fill (fill, draw, espessura etc.).
% - texto:            string desenhada ao longo do arco médio.
% --------------------------------------------------------------
\newcommand{\arcarrow}[8]{%
  % Normalização numérica (evita expansão prematura)
  \pgfmathsetmacro{\rin}{#1}
  \pgfmathsetmacro{\rmid}{#2}
  \pgfmathsetmacro{\rout}{#3}
  \pgfmathsetmacro{\astart}{#4}
  \pgfmathsetmacro{\aend}{#5}
  \pgfmathsetmacro{\atip}{#6}

  % Shape preenchido (cunha + ponta)
  \fill[#7]
    (\astart:\rin) arc (\astart:\aend:\rin)   % arco interno
    -- (\aend+\atip:\rmid)                    % ponta no raio médio
    -- (\aend:\rout) arc (\aend:\astart:\rout)% arco externo (volta)
    -- (\astart+\atip:\rmid) -- cycle;        % fecha a cunha no raio médio

  % Texto ao longo do arco médio (opcional; remova se não quiser texto interno)
  \path[
    decoration={text along path, text={|\mytextstyle|#8},
                text align=center, raise=-0.3ex},
    decorate
  ] (\astart+\atip:\rmid) arc (\astart+\atip:\aend+\atip:\rmid);
}



% ==============================================================
% --- TikZ: bibliotecas necessárias ---
\usetikzlibrary{decorations.text,positioning, shadows}
\usetikzlibrary{arrows.meta,calc} % (calc para coordenadas polares/expressões)
% --------------------------------------------------------------
% Estilo do texto desenhado AO LONGO DO ARCO (interno às setas)
% --------------------------------------------------------------
\newcommand*{\mytextstyle}{\sffamily\footnotesize\bfseries\color{black!85}}

% --------------------------------------------------------------
% MACRO PRINCIPAL: \arcarrow
% Desenha UMA seta em arco com texto curvado.
%
% Assinatura:
%   \arcarrow{rin}{rmid}{rout}{angIni}{angFim}{tip}{tikz-options}{texto}
%
% - rin, rmid, rout: raios interno, médio (texto) e externo da seta.
% - angIni, angFim:  ângulos em graus (orientação padrão do TikZ).
% - tip:              avanço (em graus) no raio médio para formar a “ponta”.
% - tikz-options:     opções passadas ao \fill (fill, draw, espessura etc.).
% - texto:            string desenhada ao longo do arco médio.
% --------------------------------------------------------------
\newcommand{\arcarrow}[8]{%
  % Normalização numérica (evita expansão prematura)
  \pgfmathsetmacro{\rin}{#1}
  \pgfmathsetmacro{\rmid}{#2}
  \pgfmathsetmacro{\rout}{#3}
  \pgfmathsetmacro{\astart}{#4}
  \pgfmathsetmacro{\aend}{#5}
  \pgfmathsetmacro{\atip}{#6}

  % Shape preenchido (cunha + ponta)
  \fill[#7]
    (\astart:\rin) arc (\astart:\aend:\rin)   % arco interno
    -- (\aend+\atip:\rmid)                    % ponta no raio médio
    -- (\aend:\rout) arc (\aend:\astart:\rout)% arco externo (volta)
    -- (\astart+\atip:\rmid) -- cycle;        % fecha a cunha no raio médio

  % Texto ao longo do arco médio (opcional; remova se não quiser texto interno)
  \path[
    decoration={text along path, text={|\mytextstyle|#8},
                text align=center, raise=-0.3ex},
    decorate
  ] (\astart+\atip:\rmid) arc (\astart+\atip:\aend+\atip:\rmid);
}



% ==============================================================
% --- TikZ: bibliotecas necessárias ---
\usetikzlibrary{decorations.text,positioning, shadows}
\usetikzlibrary{arrows.meta,calc} % (calc para coordenadas polares/expressões)
% --------------------------------------------------------------
% Estilo do texto desenhado AO LONGO DO ARCO (interno às setas)
% --------------------------------------------------------------
\newcommand*{\mytextstyle}{\sffamily\footnotesize\bfseries\color{black!85}}

% --------------------------------------------------------------
% MACRO PRINCIPAL: \arcarrow
% Desenha UMA seta em arco com texto curvado.
%
% Assinatura:
%   \arcarrow{rin}{rmid}{rout}{angIni}{angFim}{tip}{tikz-options}{texto}
%
% - rin, rmid, rout: raios interno, médio (texto) e externo da seta.
% - angIni, angFim:  ângulos em graus (orientação padrão do TikZ).
% - tip:              avanço (em graus) no raio médio para formar a “ponta”.
% - tikz-options:     opções passadas ao \fill (fill, draw, espessura etc.).
% - texto:            string desenhada ao longo do arco médio.
% --------------------------------------------------------------
\newcommand{\arcarrow}[8]{%
  % Normalização numérica (evita expansão prematura)
  \pgfmathsetmacro{\rin}{#1}
  \pgfmathsetmacro{\rmid}{#2}
  \pgfmathsetmacro{\rout}{#3}
  \pgfmathsetmacro{\astart}{#4}
  \pgfmathsetmacro{\aend}{#5}
  \pgfmathsetmacro{\atip}{#6}

  % Shape preenchido (cunha + ponta)
  \fill[#7]
    (\astart:\rin) arc (\astart:\aend:\rin)   % arco interno
    -- (\aend+\atip:\rmid)                    % ponta no raio médio
    -- (\aend:\rout) arc (\aend:\astart:\rout)% arco externo (volta)
    -- (\astart+\atip:\rmid) -- cycle;        % fecha a cunha no raio médio

  % Texto ao longo do arco médio (opcional; remova se não quiser texto interno)
  \path[
    decoration={text along path, text={|\mytextstyle|#8},
                text align=center, raise=-0.3ex},
    decorate
  ] (\astart+\atip:\rmid) arc (\astart+\atip:\aend+\atip:\rmid);
}



% ==============================================================
% --- TikZ: bibliotecas necessárias ---
\usetikzlibrary{decorations.text,positioning, shadows}
\usetikzlibrary{arrows.meta,calc} % (calc para coordenadas polares/expressões)
% --------------------------------------------------------------
% Estilo do texto desenhado AO LONGO DO ARCO (interno às setas)
% --------------------------------------------------------------
\newcommand*{\mytextstyle}{\sffamily\footnotesize\bfseries\color{black!85}}

% --------------------------------------------------------------
% MACRO PRINCIPAL: \arcarrow
% Desenha UMA seta em arco com texto curvado.
%
% Assinatura:
%   \arcarrow{rin}{rmid}{rout}{angIni}{angFim}{tip}{tikz-options}{texto}
%
% - rin, rmid, rout: raios interno, médio (texto) e externo da seta.
% - angIni, angFim:  ângulos em graus (orientação padrão do TikZ).
% - tip:              avanço (em graus) no raio médio para formar a “ponta”.
% - tikz-options:     opções passadas ao \fill (fill, draw, espessura etc.).
% - texto:            string desenhada ao longo do arco médio.
% --------------------------------------------------------------
\newcommand{\arcarrow}[8]{%
  % Normalização numérica (evita expansão prematura)
  \pgfmathsetmacro{\rin}{#1}
  \pgfmathsetmacro{\rmid}{#2}
  \pgfmathsetmacro{\rout}{#3}
  \pgfmathsetmacro{\astart}{#4}
  \pgfmathsetmacro{\aend}{#5}
  \pgfmathsetmacro{\atip}{#6}

  % Shape preenchido (cunha + ponta)
  \fill[#7]
    (\astart:\rin) arc (\astart:\aend:\rin)   % arco interno
    -- (\aend+\atip:\rmid)                    % ponta no raio médio
    -- (\aend:\rout) arc (\aend:\astart:\rout)% arco externo (volta)
    -- (\astart+\atip:\rmid) -- cycle;        % fecha a cunha no raio médio

  % Texto ao longo do arco médio (opcional; remova se não quiser texto interno)
  \path[
    decoration={text along path, text={|\mytextstyle|#8},
                text align=center, raise=-0.3ex},
    decorate
  ] (\astart+\atip:\rmid) arc (\astart+\atip:\aend+\atip:\rmid);
}



%======================================================================
% Livro: Introdução à Assimilação de Dados
% Autor: João Gerd Zell de Mattos
% Compilação recomendada: pdflatex -> biber -> pdflatex -> pdflatex
%======================================================================
\documentclass[12pt,a4paper,oneside]{book}

%----------------------------- Pacotes --------------------------------
\usepackage[utf8]{inputenc}
\usepackage[T1]{fontenc}
\usepackage[brazil]{babel}
\usepackage{lmodern}
\usepackage{microtype}
\usepackage{setspace}
\usepackage[a4paper,margin=2.5cm]{geometry}
\usepackage{graphicx}
\usepackage{caption}
\usepackage{subcaption}
\usepackage{booktabs}
\usepackage{amsmath,amssymb,amsfonts}
\usepackage{xcolor}
\usepackage{csquotes}
\usepackage{hyperref}
\usepackage{bookmark}
\usepackage{fancyhdr}
\usepackage{titlesec}
\usepackage{epigraph}
\usepackage[acronym,toc]{glossaries}
\usepackage{makeidx}
\usepackage[backend=biber,style=authoryear,maxbibnames=10,uniquename=false]{biblatex}

% Bibliografia (arquivo externo .bib)
\addbibresource{references.bib}

%----------------------- Metadados do documento -----------------------
\title{Introdução à Assimilação de Dados}
\author{João Gerd Zell de Mattos}
\date{\today}

%------------------------ Aparência e estilos -------------------------
\onehalfspacing
\definecolor{brandA}{HTML}{0B5FA5} % azul
\definecolor{brandB}{HTML}{F28C28} % laranja
\definecolor{ink}{HTML}{1F2937}     % cinza escuro

\hypersetup{
  colorlinks=true,
  linkcolor=brandA,
  citecolor=brandB,
  urlcolor=brandA,
  pdftitle={Introdução à Assimilação de Dados},
  pdfauthor={João Gerd Zell de Mattos},
  pdfsubject={Assimilação de Dados},
  pdfkeywords={assimilação de dados, análise objetiva, filtros, kalman, var, enkf}
}

\pagestyle{fancy}
\fancyhf{}
\lhead{\leftmark}
\rfoot{\thepage}

% Títulos mais limpos
\titleformat{\chapter}[display]
  {\bfseries\Huge\color{ink}}
  {\filright\Large\color{brandB}\thechapter}
  {1ex}
  {\titlerule[0.8pt]\vspace{1ex}\filright}
  [\vspace{1ex}\titlerule]

\titleformat{\section}
  {\bfseries\Large\color{ink}}{\thesection}{0.8em}{}

% Epígrafe mais elegante
\setlength{\epigraphwidth}{0.7\textwidth}
\setlength{\epigraphrule}{0pt}

% Glossário e índice
\makeglossaries
\newacronym{ad}{AD}{Assimilação de Dados}
\newacronym{obana}{OBAN}{Análise Objetiva}
\newacronym{enkf}{EnKF}{Ensemble Kalman Filter}
\newacronym{var}{VAR}{Método Variacional}
\makeindex

%------------------------------ Capa ----------------------------------
\newcommand{\CustomTitlePage}{
\begin{titlepage}
  \centering
  \vspace*{1cm}
  % Faixa colorida
  \fcolorbox{white}{brandA}{%
    \begin{minipage}{0.95\linewidth}
      \vspace*{1.6cm}
      \centering
      {\bfseries\fontsize{28pt}{30pt}\selectfont\color{white} Introdução à Assimilação de Dados\par}
      \vspace{0.6cm}
      {\Large\color{white} Fundamentos de interpolação, análise objetiva e métodos modernos\par}
      \vspace{1.6cm}
    \end{minipage}
  }
  \vfill
  {\Large\color{ink} \textbf{João Gerd Zell de Mattos}\par}
  \vspace{0.25cm}
  {\large\color{ink} \today\par}
  \vspace{1.5cm}
  \begin{center}
    \rule{0.35\linewidth}{0.6pt}\par
    \vspace{0.2cm}
    {\small\color{ink} (rascunho / edição do autor)}
  \end{center}
\end{titlepage}
}

%======================================================================
\begin{document}
\CustomTitlePage
\frontmatter

%---------------------------- Dedicatória -----------------------------
\chapter*{Dedicatória}
\addcontentsline{toc}{chapter}{Dedicatória}
\vspace*{2cm}
\begin{flushright}
\emph{A todos que acreditam que unir ciência, computação e observações\\
pode aproximar a previsão da realidade.}
\end{flushright}

%----------------------------- Epígrafe -------------------------------
\chapter*{Epígrafe}
\addcontentsline{toc}{chapter}{Epígrafe}
\epigraph{Entre o que o modelo prevê e o que o instrumento observa,\\
buscamos a melhor estimativa possível.}{\textit{Autor desconhecido}}

%-------------------------- Agradecimentos ----------------------------
\chapter*{Agradecimentos}
\addcontentsline{toc}{chapter}{Agradecimentos}
Agradeço aos colegas, estudantes e parceiros institucionais que, direta ou indiretamente, contribuíram com discussões, dados, código e ideias. Aos leitores, pela curiosidade e paciência necessárias para trilhar os fundamentos e as fronteiras da \gls{ad}.

%------------------------------ Prefácio ------------------------------
\chapter*{Prefácio}
\addcontentsline{toc}{chapter}{Prefácio}
Este livro nasce com a intenção de apresentar a \gls{ad} a partir de um caminho intuitivo: da interpolação clássica aos métodos modernos que ponderam informações por seus erros e covariâncias. A obra organiza conceitos essenciais, exemplos práticos e ponte com sistemas operacionais, servindo tanto como porta de entrada quanto como mapa para estudos mais avançados.

% Sumário
\tableofcontents

\mainmatter

%======================================================================
% PARTE I
%======================================================================
\part{O caminho até a assimilação de dados}

\chapter{A arte de combinar informações}
\noindent\textbf{Resumo:} Este capítulo introduz a ideia fundamental da \gls{ad} como a arte de combinar diferentes fontes de informação — o modelo numérico e as observações — para aproximar-se do estado real do sistema. Estabelece o paralelo entre interpolação e estimativa estatística com pesos baseados em erro.
\section{Entre o modelo e a realidade}
\section{A interpolação como estimativa ponderada}
\section{A transição para pesos baseados em erro}

\chapter{Interpolação: o ponto de partida}
\noindent\textbf{Resumo:} Fundamentos de interpolação (linear, polinomial e afins), noções de continuidade/suavidade e limitações práticas em presença de ruído, lacunas e extrapolação.
\section{Interpolação linear e polinomial}
\section{Limitações e desafios práticos}
\section{Aplicações meteorológicas simples}

\chapter{Ajuste por mínimos quadrados}
\noindent\textbf{Resumo:} Do ajuste exato ao ótimo: resíduos, função custo e a ideia de melhor estimativa global quando os dados são ruidosos e inconsistentes.
\section{Da interpolação exata ao ajuste ótimo}
\section{Erros, resíduos e função custo}
\section{A noção de melhor estimativa}

\chapter{Ajustes em duas dimensões e função de suavização}
\noindent\textbf{Resumo:} Extensão espacial em 2D, construção de campos em grade, regularização/suavização e compromissos entre fidelidade aos dados e coerência espacial.
\section{Extensão espacial dos ajustes}
\section{Suavização e regularização}
\section{A construção de campos meteorológicos em grade}

\chapter{Análise objetiva e o nascimento da assimilação de dados}
\noindent\textbf{Resumo:} Transição para a análise objetiva (\gls{obana}): Cressman, Barnes e Sucessive Correction, resposta do filtro e papel das escalas. Ponte para a \gls{ad} moderna.
\section{O conceito de análise objetiva (OBAN)}
\section{Sucessive Correction, Cressman e Barnes}
\section{Filtros espaciais e resposta da análise}

%======================================================================
% PARTE II
%======================================================================
\part{A assimilação de dados moderna}

\chapter{Formulação estatística da assimilação de dados}
\noindent\textbf{Resumo:} Definição de background, observações e análise; papel das matrizes de erro (B e R) e interpretação da análise como combinação ótima estatisticamente fundamentada.
\section{Background, observações e análise}
\section{As matrizes de erro e seus papéis}
\section{A análise como combinação ótima}

\chapter{O filtro de Kalman e suas variações}
\noindent\textbf{Resumo:} Filosofia sequencial do Kalman, extensões variacionais (3D/4D-\gls{var}) e filtros de conjunto (\gls{enkf}), ressaltando custo computacional, representação de erro e não linearidades.
\section{Introdução ao filtro de Kalman}
\section{Extensão para 3DVAR e 4DVAR}
\section{Filtros de conjunto (EnKF e variantes)}

\chapter{Sistemas operacionais e exemplos reais}
\noindent\textbf{Resumo:} Visão prática: sistemas GSI, JEDI, LETKF; o ciclo de assimilação e exemplos com observações de superfície, sondagens e satélite.
\section{Sistemas de assimilação: GSI, JEDI, LETKF}
\section{O ciclo de assimilação}
\section{Casos práticos com dados de satélite e superfície}

\chapter{O futuro da assimilação de dados}
\noindent\textbf{Resumo:} Tendências: métodos híbridos, aprendizado de máquina, sistemas acoplados e desafios em regiões tropicais; novas plataformas observacionais.
\section{Métodos híbridos e aprendizado de máquina}
\section{Assimilação no sistema terrestre}
\section{Desafios tropicais e novas observações}

%======================================================================
% Materiais de apoio
%======================================================================
\backmatter

% Glossário e Siglas
\printglossary[type=\acronymtype,title={Lista de Siglas}]
\printglossary[title={Glossário}]

% Bibliografia
\printbibliography[title={Referências}]

% Índice remissivo
\printindex

% Apêndices
\appendix
\chapter{Apêndice A: Notação e símbolos}
\chapter{Apêndice B: Exercícios comentados}

\end{document}
